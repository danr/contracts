\documentclass[preprint,nocopyrightspace,draft]{sigplanconf}

\usepackage{hcc-techreport}

\begin{document} 
\preprintfooter{\textbf{--- DRAFT ---}}

%%% Extra definitions -- move to hcc-techreport at some point (carefully to not break that!)
\newcommand{\Ct}{{\tt C}}
\newcommand{\CF}{{\tt CF}}
\newcommand{\True}{\textit{True}}
\newcommand{\False}{\textit{False}}
\newcommand{\Bool}{\mathop{Bool}}
\newcommand{\ys}{\ol{y}}
\newcommand{\Th}[2]{{\cal T}_{#1,#2}}
\newcommand{\Ecf}{\textsc{Ecf}}
\newcommand{\oln}[2]{\ol{#1}^{#2}}
\newcommand{\tmar}[2]{\mathop{tmar}_{#1}(#2)}
\newcommand{\tyar}[2]{\mathop{tyar}_{#1}(#2)}
\newcommand{\ar}{n}
\newcommand{\lcf}[1]{\textsf{cf}(#1)}
\newcommand{\lcfZ}{\textsf{cf}}
\newcommand{\lncf}[1]{\neg\textsf{cf}(#1)}
\newcommand{\unr}{\mathop{unr}}
\newcommand{\bad}{\mathop{bad}}
\newcommand{\sel}[2]{\mathop{sel\_#1\!\_#2}}
\newcommand{\ctrans}[3]{{\cal C}\{\!\!\{#3\}\!\!\}_{#2}}
\newcommand{\etrans}[3]{{\cal E}\{\!\!\{#3\}\!\!\}_{#2}}
\newcommand{\utrans}[3]{{\cal U}\{\!\!\{#3\}\!\!\}_{#2}}
\newcommand{\dtrans}[2]{{\cal D}\{\!\!\{#2\}\!\!\}}
\newcommand{\Dtrans}[2]{{\cal D}\{\!\!\{#2\}\!\!\}}
%% Gadgets of domain theory 
\newcommand{\roll}{\mathsf{roll}}
\newcommand{\unroll}{\mathsf{unroll}}
\newcommand{\bind}{\mathsf{bind}}
\newcommand{\ret}{\mathsf{ret}}
\newcommand{\dlambda}{\mathsf{\lambda}}
\newcommand{\curry}{\mathsf{curry}}
\newcommand{\eval}{\mathsf{eval}}
\newcommand{\uncurry}{\mathsf{incurry}}
\newcommand{\dapp}{\mathsf{app}}
\newcommand{\inj}[1]{\mathsf{inj}_{#1}}
\newcommand{\unitcpo}{{\sf{\bf 1}}}
\newcommand{\VarCpo}{\textit{Var}}
\newcommand{\FVarCpo}{\textit{FunVar}}
\newcommand{\interp}[3]{[\![#1]\!]_{\langle {#2},{#3}\rangle}}
\newcommand{\dbrace}[1]{[\![#1]\!]}
\newcommand{\linterp}[1]{{\cal I}(#1)}
\newcommand{\lassign}[1]{\mu(#1)}
\newcommand{\elab}[1]{\rightsquigarrow \formula{#1}}
\newcommand{\Fcf}{F_{\lcfZ}} 
\newcommand{\definable}[1]{{\mathop{def}}(#1)}
\newcommand{\curly}{\rightsquigarrow}
\newcommand{\Min}{\cal M}
\newcommand{\mlinterp}[1]{{\cal I}^\mu(#1)}

\renewcommand{\Th}{\cal T}

\title{HALO: Haskell to Logic through Denotational Semantics}
%% \subtitle{A new approach to static contract checking for higher-order lazy programs}

\authorinfo{Dan Ros\'{e}n \\ Koen Claessen}
           {Chalmers University}{}
\authorinfo{Dimitrios Vytiniotis \\ Simon Peyton Jones}
           {Microsoft Research}{}
\authorinfo{Nathan Collins}
           {Portland State University}{}
\maketitle
\makeatactive

\begin{abstract}
Despite the benefits of strong types and purity for reasoning about
programs, many bugs remain inside well-typed and purely functional code, 
and programmers often introduce assertions to ensure that their code is crash-free. 
In this work we allow programmers to express assertions as 
contracts, whose validity we can check statically. 
We contribute a novel translation of programs and 
contracts written in the same lazy, higher-order, and recursive language to first-order logic, 
justified by denotational semantics. This translation enables a simple and distinctly different
methodology for static contract checking compared to previous work: instead of wrapping
and symbolic execution we harness purity to directly
encode the denotational semantics of programs and contracts in first-order logic, and
invoke a first-order theorem prover. For this paper we focus on the translation, but in 
addition we have implemented a static contract checker for Haskell programs and evaluated 
the practicality of this approach on many examples, including lazy and higher-order programs.
\end{abstract}



\section{Introduction}\label{s:intro}
  Haskell programmers enjoy the benefits of strong static types and purity: 
static types eliminate many bugs early on in the development cycle, and purity 
simplifies equational reasoning about programs. Despite these benefits, however, 
bugs may still remain inside purely functional code and programs often
crash if applied to the wrong arguments. 

Consider this Haskell definition:
\begin{code}
  f xs = head (reverse (True : xs))
  g xs = head (reverse xs)
\end{code}
Both @f@ and @g@ are well typed (and hence do not ``go wrong'' in Milner's 
sense), but @g@ can crash (when applied to the empty list), whereas @f@ cannot.
To distinguish the two we need reasoning that goes well beyond 
that typically embodied in a standard type system.

Many variations of {\em dependent type systems}~\cite{norell:thesis,Xi:2007:DMA:1230756.1230759,fstar} or 
{\em refinement type systems}~\cite{Rondon:2008:LT:1375581.1375602,Knowles+:sage}
have been proposed to address this problem, each offering different degrees of 
expressiveness or automation. 
Another line of work aiming to address this problem, studied by many researchers
as well~\cite{Findler:2002:CHF:581478.581484,Blume:2006:SCM:1166013.1166016,Knowles+:sage,Siek06gradualtyping,Wadler:2009:WPC:1532974.1532976}, allows programmers to annotate 
functions with {\em contracts}, which are forms of behavioural specifications.
For instance, we might write the following contract for 
@reverse@: 
\[ @reverse@ \in (xs : \CF) -> \{ ys \mid @null@\;xs\; @<=>@\; @null@\;ys \}  \] 
%% \begin{code}
%%   reverse ::: xs:CF -> { ys | null xs <=> null ys }
%% \end{code}
This contract annotation asserts that if @reverse@ is applied to a 
crash-free (@CF@) argument list @xs@ then the result @ys@ will be empty (@null@) 
if and only if @xs@ is empty. What is a crash-free argument? 
Since we are using lazy semantics, a list could possibly contain cons-cells that yield 
errors only when evaluated, and the @CF@ precondition asserts that the input list 
is not one of those.

Notice also that @null@ and @<=>@ are just ordinary Haskell functions, perhaps
written by the programmer, even though they appear inside contracts.
With this in hand we might hope to prove that @f@ satisfies the contract
\[ @f@ \in \CF -> \CF \] 
But how do we verify that @reverse@ and @f@ satisfy the claimed
contracts? Contracts are often tested dynamically, but 
our plan here is different: we want to verify contracts \emph{statically} 
and \emph{automatically}. 

It should be clear that there is a good deal of logical reasoning to do,
and a now-popular approach is to hand off the task to an off-the-shelf theorem
prover such as Z3~\cite{z3citation} or Vampire~\cite{vampire}, or search for 
counterexamples with a finite model finder~\cite{paradox}.
With that in mind, we make the following new contributions:

\begin{itemize}
  \item We give a translation of Haskell programs to first-order logic (FOL) theories. 
        It turns out that lazy programs (as opposed to
        strict programs!) have a very natural translation into first-order logic
        (Section~\ref{ssect:trans-fol}).
  \item We give a translation of contracts to FOL formulae, and an axiomatisation of 
        the language semantics in FOL 
        (Section~\ref{s:contracts-fol}).
  \item Our main contribution is to show that if we can prove the formula 
        that arises from a contract translation 
        for a given program, then the program does indeed satisfy this contract. Our proof
        uses the novel idea of employing the denotational 
        semantics as a first-order model (Section~\ref{ssect:denot}).
  \item We show how to use this translation in practice for static contract checking with
        a SAT-solver (Section~\ref{sect:soundness}), 
        and how to prove goals by induction (Section~\ref{sect:extensions}).
\end{itemize}

We consider this work to be the first step towards practical contract checking 
for Haskell programs, that lays out the theoretical foundations for further engineering 
and experimentation. Nevertheless, we have already implemented a prototype for Haskell 
programs that uses GHC as a front-end. We have evaluated the practicality of our approach 
on many examples, including lazy and higher-order programs, and goals that require 
induction. We report this initial encouraging evaluation in 
Section~\ref{sect:implementation}. 

To our knowledge no-one has previously presented a translation of lazy higher-order programs to 
first-order logic in a provably sound way with respect to a denotational
semantics. Furthermore, our approach to static contract checking is 
distinctly different to previous work: instead of wrapping and 
symbolic execution~\cite{xu+:contracts,Xu:2012:HCC:2103746.2103767}, 
we harness purity and laziness to directly use the denotational semantics
of programs and contracts and discharge the obligations with a SAT-solver, side-stepping
the wrapping process. Instead of verification condition generation by pushing
pre- and post- conditions through a program, we directly ask a theorem prover to prove 
a contract for the FOL encoding of a program. 
We further discuss similarities and differences compared to related work in Section~\ref{sect:related}.

%% \newpage 
%%   \item The translation
%% For this paper we focus on the translation, but to substantiate the practicality 
        
        
%% \end{itemize} 


%% \begin{itemize}
%% \item We show how to translate Haskell terms
%% into First Order Logic (FOL) (Section~\ref{ssect:trans-fol}).  
%% It may appear surprising that this 
%% is even possible, since Haskell is a higher order language.  Although
%% the basic idea of the translation is folklore in the community,
%% we believe that this paper is the first to explain it explicitly.

%% \item We also show how to translate \emph{contracts} into FOL
%%       (Section~\ref{s:contracts-fol}), 
%%       a translation that is rather less obvious.

%% \item We give a proof based on denotational semantics 
%% that if the FOL prover discharges a 
%% suitable theorem about the translated Haskell term and contract, 
%% then indeed the original Haskell term satisfies that contract (Section~\ref{s:xxx}).

%% %\item It is one thing to make a sound translation, and quite another
%% %to produce FOL terms that the FOL prover can actually prove anything
%% %about --- a common experience is that it goes out to lunch instead.  We
%% %describe a number of techniques that dramatically improve
%% %theorem-proving times, moving them from infeasible to feasible (Section\ref{xxx}).

%% \item For this paper we focus on the
%% translation, but we have also implemented a static contract checker
%% for Haskell itself, by using GHC as a front end.  We have evaluated
%% the practicality of this approach on many examples, including lazy and
%% higher-order programs, as we describe in Section~\ref{xxx}.  \spj{I'd like
%% to say something more substantial here.}
%% \end{itemize}





















\section{Checking Haskell contracts in practice}\label{s:examples}
  TODO

\subsection{A high-level overview of the system}\label{ssect:schematic}



\section{A higher-order lazy language}\label{sect:language}
  To formalise the ideas behind our implementation, we define a
tiny source language $\cal L$:
polymorphic, higher-order, call-by-name $\lambda$-calculus with
algebraic datatypes, pattern matching, and recursion.
Our actual implementation treats all of Haskell, by using GHC as a front
end to translate Haskell into language $\cal L$.

\subsection{Syntax of $\cal L$} \label{s:syntax}

\begin{figure}
\[\begin{array}{l}
\begin{array}{lrll}
\multicolumn{3}{l}{\text{Programs, definitions, and expressions}} \\
P   & ::= & d_1 \ldots d_n \\
d   & ::= & f\; \ol{a} \; \ol{(x\!:\!\tau)} = u \\
u   & ::= & e \; \mid \; @case@\;e\;@of@\;\ol{K\;\ol{y} -> e}
\end{array}
\\
\begin{array}{lrll}
e  & ::=  & x            & \text{Variables} \\
   & \mid & f[\ol{\tau}] & \text{Function variables} \\
   & \mid & K[\ol{\tau}](\ol{e}) & \text{Data constructors (fully applied)} \\
   & \mid & e\;e         & \text{Applications} \\
   & \mid & @BAD@        & \text{Runtime error} \\
\end{array}\\ \\
\begin{array}{lrll}
\multicolumn{3}{l}{\text{Syntax of closed values}} \\
 v,w & ::= & K^\ar[\ol{\tau}](\oln{e}{\ar}) \;\mid\; f^\ar[\ol{\tau}]\;\oln{e}{m < \ar} \;\mid\; @BAD@ \\ \\
\end{array}
\\
\begin{array}{lrll}
\multicolumn{3}{l}{\text{Contracts}} \\
 \Ct & ::=  & \{ x \mid e \}        & \text{Base contracts}  \\
     & \mid &  (x : \Ct_1) -> \Ct_2      & \text{Arrow contracts} \\
     & \mid & \Ct_1 \& \Ct_2             & \text{Conjunctions}   \\
     & \mid & \CF                        & \text{Crash-freedom}   \\
\end{array}
\\ \\
\begin{array}{lrll}
\multicolumn{3}{l}{\text{Types}} \\
\tau,\sigma & ::=  & T\;\taus & \text{Datatypes} \\
            & \mid & a \mid \tau -> \tau
\end{array}
\\ \\
\begin{array}{lrll}
\multicolumn{3}{l}{\text{Type environments and signatures}} \\
\Gamma & ::=  & \cdot \mid \Gamma,x \\
\Delta & ::=  & \cdot \mid \Delta,a \mid \Delta,x{:}\tau \\
\Sigma & ::=  & \cdot \mid \Sigma,T{:}n \mid \Sigma,f{:}\forall\ol{a} @.@ \tau \mid \Sigma,K^{\ar}{:}\forall\ol{a} @.@ \oln{\tau}{\ar} -> @T@\;\as
\end{array}
\\ \\
\begin{array}{lrll}
\multicolumn{3}{l}{\text{Auxiliary functions}} \\
%% constrs(\Sigma,T) & = & \{ K \mid (K{:}\forall \as @.@ \taus -> T\;\as) \in \Sigma \} \\
(\cdot)^{-}            & = & \cdot \\
(\Delta,a)^{-}         & = & \Delta^{-} \\
(\Delta,(x{:}\tau)^{-} & = & \Delta^{-},x
%% \tyar{D}{f} & = & n & \\
%%             & \multicolumn{3}{l}{\text{when}\; (f |-> \Lambda\oln{a}{n} @.@ \lambda\ol{x{:}\tau} @.@ u) \in D} \\
%% \tmar{D}{f} & = & n & \\
%%             & \multicolumn{3}{l}{\text{when}\; (f |-> \Lambda\ol{a} @.@ \lambda\oln{x{:}\tau}{n} @.@ u) \in D}
\end{array}
\end{array}\]
\caption{Syntax of $\cal L$ and its contracts}\label{fig:syntax}
\end{figure}

Figure~\ref{fig:syntax} presents the syntax of $\cal L$.  A program
$P$ consists of a set of recursive function definitions $d_1 \ldots
d_n$. Each definition has a left hand side that binds its type-variable and
term-variable parameters;
if $f$ has $n$ term-variable parameters we say that
it has arity $n$, and sometimes write it $f^n$.
The right hand side $u$ of a definition is either a @case@ expression or a
@case@-free expression $e$.  A @case@-free expression consists of
variables $x$, function variables $f[\taus]$ fully applied to their
type arguments, applications $e_1\;e_2$, data constructor applications
$K[\taus](\ol{e})$, as well as the special value @BAD@, which will be
used to model failure as a throwable error term. As a notation, we use
$\oln{x}{n}$ for sequences of elements of size $n$. When $n$ is
omitted $\ol{x}$ has a size which is implied by the context or is not
interesting.

Our language embodies several convenient syntactic constraints: (i)~$\lambda$
abstractions occur only at the top-level, (ii)~@case@-expressions can
only immediately follow a function definition, and (iii) constructors
are fully applied.  These constraints do not restrict expressiveness;
lambda-lifting, @case@-lifting, and eta-expansion respectively can
easily transform a program with nested constructs and
partially-applied constructors into our restricted form.  Indeed our
prototype relies on existing implementation of similar transformations
from the GHC-as-a-library API. However this simpler language is
instead extremely convenient for the translation of programs to
first-order logic.

Figure~\ref{fig:syntax} embodies one other inessential choice in order to
facilitate the formalisation and implementation: we assume that
functions have arity at least one (disallowing CAF's).  \spj{Disallowing CAFs is quite significant.
We need to say more about why.} \dv{It is not essential -- it was in some previous version of a proof. I will fix this.}

$\cal L$ is an explicitly-typed language, and we assume the existence
of a typing relation $\Sigma |- P$, which checks that a program
conforms to the definitions in the signature $\Sigma$. A signature
$\Sigma$ (Figure~\ref{fig:syntax}) records the declared data types,
data constructors and types of functions in the program $P$. The
well-formedness of expressions is checked with a typing relation
$\Sigma;\Delta |- u : \tau$, where $\Delta$ is a typing environment,
also in Figure~\ref{fig:syntax}.  We do not give the details of the
typing relation since it is standard.
Our technical development and analysis in the following sections
assume that programs have been checked for type errors.

One property that we require from the typing relation is
that it {\em asserts the exhaustiveness of pattern matches}. In an
{\em actual} source language programmers may omit pattern matches but
here we will assume that all pattern matches are
exhaustive. Originally-incomplete cases have been completed to return
the crashing term @BAD@. For instance, the program:
\begin{code}
head :: [a] -> [a]
head (x:xs) = x
\end{code}
will be represented in our language as:
\[\begin{array}{l}
   @head@\; a\; (x{:}[a]) = @case@\;x\;@of@ \{ [] -> @BAD@ ; (x:xs) -> x \}
\end{array}\]
We also assume that the standard Haskell function @error@ simply invokes @BAD@, thus:
\begin{code}
  error :: String -> a
  error s = BAD
\end{code}
In our context, @BAD@ is our way to saying what it means for a program to ``go wrong'',
and verification amounts to proving that a program cannot invoke @BAD@.

The syntax of closed values is also given in Figure~\ref{fig:syntax}. Since we do not
have arbitrary $\lambda$-abstractions, values can only be partial function applications
$f^\ar[\ol{\tau}]\;\oln{e}{m < \ar}$, data constructor applications $K[\tau](\ol{e})$,
and the error term @BAD@.

\subsection{Operational semantics of $\cal L$}

\spj{Why do we give an operational as well as denotational semantics?}
The big-step operational semantics of our language is given in
Figure~\ref{fig:opsem}, which contains no surprises. One interesting
detail of big-step semantics is that they do not distinguish between non-termination
and ``getting stuck'', meaning that if $P \not|- e \Downarrow v$ then $e$ could either diverge or its
evaluation could get stuck. We return to this convenient for our purposes form of operational
semantics later. \spj{Where later? Also this sentence is hard to parse; indeed I'm not quite
sure what it means.}
\begin{figure}
\[\begin{array}{c}
\ruleform{P |- u \Downarrow v} \\
\prooftree
\begin{array}{c} \ \\
\end{array}
-------------------------------------{EVal}
P |- v \Downarrow v
~~~~~
\begin{array}{c}
(f \;\ol{a}\;\oln{(x{:}\tau)}{m} = u) \in P \\
P |- u[\ol{\tau}/\ol{a}][\ol{e}/\ol{x}] \Downarrow v
\end{array}
-------------------------------------{EFun}
P |- f[\ol{\tau}]\;\oln{e}{m} \Downarrow v
~~~~~
\begin{array}{c}
P |- e_1 \Downarrow v_1 \\
P |- v_1\;e_2 \Downarrow w
\end{array}
------------------------------------------------{EApp}
P |- e_1\;e_2 \Downarrow w
~~~~~
\begin{array}{c}
P |- e_1 \Downarrow @BAD@
\end{array}
------------------------------------------------{EBadApp}
P |- e_1\;e_2 \Downarrow @BAD@
~~~~~
%% \endprooftree \\ \\
%% \ruleform{P |- u \Downarrow v} \\ \\
%% \prooftree
%% P |- e \Downarrow v
%% -------------------------------------{EUTm}
%% P |- e \Downarrow v
%% ~~~~
\begin{array}{c}
P |- e \Downarrow K_i[\ol{\sigma}_i](\ol{e}_i) \quad
P |- e'_i[\ol{e}_i/\ol{y}_i] \Downarrow w
\end{array}
------------------------------------{ECase}
P |- @case@\;e\;@of@\;\ol{K\;\ol{y} -> e'} \Downarrow w
~~~~~
\begin{array}{c}
P |- e \Downarrow @BAD@ \\
\end{array}
------------------------------------{EBadCase}
P |- @case@\;e\;@of@\;\ol{K\;\ol{y} -> e'} \Downarrow @BAD@
%% \begin{array}{c}
%% (f |-> \Lambda\ol{a} @.@ \lambda\oln{x{:}\tau}{m} @.@ @case@\;e\;@of@\;\ol{K\;\ol{y} -> e'}) \in D \\
%% D |- e[\ol{\tau}/\ol{a}][\ol{e}/\ol{x}] \Downarrow @BAD@ \\
%% \end{array}
%% -------------------------------------{EBadCase}
%% D |- f[\ol{\tau}]\;\oln{e}{m} \Downarrow @BAD@
\endprooftree
\end{array}\]
\caption{Operational semantics of $\cal L$}\label{fig:opsem}
\end{figure}
%% We can state some standard properties of the typing and evaluation relation.
%% \begin{lemma}[Subject reduction]
%% Assume $\Sigma |- P$ and $\Sigma;\cdot |- e : \tau$
%% If $P |- e \Downarrow w$ then $P |- value(w)$ and $\Sigma;\cdot |- w : \tau$.
%% \end{lemma}
The operational semantics of Figure~\ref{fig:opsem} has the possibility of non-determinism because
of the overlapping of several rules for applications. But that is not a problem, as we can prove that evaluation
is deterministic using the following two lemmas.
\begin{lemma}[Value determinacy]
If $\Sigma;\cdot |- v : \tau$ and
$\Sigma |- P$ and $P |- v \Downarrow w$ then $ v = w $.
\end{lemma}
\begin{lemma}[Determinacy of evaluation]
If $\Sigma;\cdot |- e : \tau$ and
$\Sigma |- P$ and $\Sigma;\cdot |- e \Downarrow v_1$ and $\Sigma;\cdot |- e \Downarrow v_2$ then
$v_1 = v_2$.
\end{lemma}
Finally, big-step soundness asserts that an expression that evaluates results in a
well-typed value.
\begin{lemma}[Big-step soundness]
If $\Sigma;\cdot |- e : \tau$ and
$\Sigma |- P$ and $\Sigma;\cdot |- e \Downarrow v$ then $\Sigma;\cdot |- v : \tau$.
\end{lemma}

%% Figure~\ref{fig:syntax} also presents the syntax of {\em contracts} to
%% keep all essential definitions grouped together, but in this section
%% we will only focus on the language and its semantics. In
%% Section~\ref{sect:contracts} we return to the syntax and semantics of
%% contracts.

%% \subsection{Denotational semantics}\label{ssect:denot}

%% Having presented the language $\cal L$ and its operational semantics, we now return to
%% our roadmap which is to axiomatize and use the denotational semantics of programs to
%% perform static contract checking. In what follows we will assume a program $P$,
%% well-formed in a signature $\Sigma$, so that $\Sigma |- P$. Most of what follows is
%% an adaptation of folklore techniques to our setting and there are no
%% surprises -- we refer the reader to~\cite{winskel} or~\cite{benton+:coq-domains}
%% for a short and modern exposition of the standard methodology.

%% Given a signature $\Sigma$ we define a strict bi-functor $F$ on complete partial
%% orders (cpos), below:
%% %% For a well-formed signature $\Sigma$, we define the strict bi-functor on cpos, below,
%% %% assuming that $K_1\ldots K_k$ are all the constructors in $\Sigma$:
%% \[\begin{array}{lclll}
%%   F(D^{-},D^{+}) & = & ( \quad{\prod_{\ar_1}{D^{+}}} & K_1^{\ar_1} \in \Sigma \\
%%                & + & \;\quad\ldots                    & \ldots \\
%%                & + & \;\quad{\prod_{\ar_k}{D^{+}}} & K_k^{\ar_k} \in \Sigma \\
%%                & + & \;\quad(D^{-} =>_c D^{+}) \\
%%                & + & \;\quad\unitcpo_{bad} \quad )_{\bot}
%% \end{array}\]
%% \spj{Why bi-functor? Why not just functor?}%% \dv{I am using Pitts' and Nick Benton's formulation and this
%% %% thing needs to be a bifunctor. It is not a functor if you have contra-varianance.}
%% The bi-functor $F$ is the lifting of a big sum: that sum consists of
%% (i) products, one for each possible constructor (even across different data types), (ii) the continuous
%% function space from $D^{-}$ to $D^{+}$, and (iii) a unit cpo to denote @BAD@ values.
%% The notation $\prod_{n}{D}$ abbreviates $n$-ary products of cpos (the unit cpo $\unitcpo$ if $n = 0$).
%% The product and sum constructions are standard, but note that we use their non-strict versions.
%% The notation $C =>_c D$ denotes the cpo
%% induced by the space of continuous functions from the cpo $C$ to the cpo $D$. We use
%% the notation $\unitcpo_{bad}$ to
%% denote a single-element cpo -- the $bad$ subscript is just there for readability.
%% The notation $D_\bot$ is {\em lifting}, which is a monad, equipped with the following two continuous
%% functions.
%% \[\begin{array}{l}
%%    \ret   : D =>_c D_\bot \\
%%    \bind_{f : D =>_c E_\bot} : D_\bot =>_c E_\bot
%% \end{array}\]
%% with the obvious definitions.

%% Observe that we have dropped all type information from the source language. The elements of the products
%% corresponding to data constructors are simply $D^{+}$ (instead of more a precise description from type
%% information) and the return types of data constructors are similarly ignored. This is not to say that
%% a more type-rich denotational semantics is not possible (or desirable even) but this simple denotational
%% semantics turns out to be sufficient for formalization and verification.

%% %% for $\lambda$-abstractions and @BAD@. Observe that we have
%% %% Moreover, the following continuous operations are defined:
%% %% \[\begin{array}{l}
%% %%    \curry_{f : D\times E =>_c F} : D =>_c (E =>_C F) \\
%% %%    \eval : (E =>_c D)\times E =>_c D
%% %% \end{array}\]
%% %% for any cpos $D, E, F$.

%% \spj{What is ``the equation induced by F''?  Maybe $$D_{\infty} = F( D_{\infty}, D_{\infty})$$?
%% Let's write it down.}
%% Using the standard {\em embedding-projection} pairs methodology we can show the following.
%% \begin{lemma}\label{lem:rec-solution}
%% There exists a solution to the domain-recursive equation induced by $F$, call it $D_{\infty}$.
%% Moreover, let a value-domain: $V_{\infty}$ be defined as:
%%     \[V_{\infty} = \begin{array}[t]{ll}
%%              \quad\;{\prod_{\ar_1}{D_{\infty}}} & K_1^{\ar_1} \in \Sigma \\
%%              \; + \;\ldots                    & \ldots \\
%%              \; + \;{\prod_{\ar_k}{D_{\infty}}} & K_k^{\ar_k} \in \Sigma \\
%%              \; + \;(D_{\infty} =>_c D_{\infty}) \\
%%              \; + \;\unitcpo_{bad} \quad
%%     \end{array}\]
%% The following continuous functions also exist, each being the inverse of the
%% other (i.e. composing to the identity function on the corresponding cpo):
%% \[\begin{array}{l}
%%   \roll : (V_{\infty})_\bot =>_c D_{\infty} \\
%%   \unroll : D_{\infty} =>_c (V_{\infty})_\bot
%% \end{array}\]
%% \end{lemma}


%% \paragraph{Definability of application}
%% We may now {\em define} application $\dapp : D_\infty \times D_\infty =>_c D_\infty$ as a continuous
%% function:
%% {\setlength{\arraycolsep}{2pt}
%% \[\begin{array}{rcll}
%%    \dapp & = & \multicolumn{2}{l}{\dlambda d @.@ \roll(\bind_g (\unroll (\pi_1(d))))} \\
%%    \text{ where } g & = &  [ & \bot : \prod_{\ar_1}{D_\infty} =>_c D_\infty =>_c (V_\infty)_\bot \\
%%                     &   &  , & \ldots \\
%%                     &   &  , & \bot : \prod_{\ar_k}{D_\infty} =>_c D_\infty =>_c (V_\infty)_\bot \\
%%                     &   &  , & \dlambda d' @.@ \unroll(d'(\pi_2(d))) \\
%%                     &   &  , & \dlambda b @.@ \dlambda d. \ret(\inj{bad}(b))\hspace{2pt} ]
%% \end{array}\]}%
%% Informally, what does the $\dapp$ combinator express? If the first
%% argument is $\bot$ it will return $\bot$, if the first argument is an
%% injection $\inj{bad}$ it will return the same, if it corresponds to a
%% data constructor it will return $\bot$. Finally, if the first element
%% is indeed a function, it will apply it.
%% \spj{Why not use $\dapp(fun, arg) = ...$ rather than all this stuff with angle brackets and pis.
%% That just adds clutter.}
%% As a notational convention, we have used notation $\langle , \rangle$ to introduce pairs and $[\ldots]$ to
%% eliminate ($n$-ary) sums.
%% \spj{I don't understand the $[\ldots]$ notation.}
%% The projections $\pi_1$ and $\pi_2$ are the obvious continuous projections from the
%% binary product space of $D_{\infty}$. We use notation $\inj{K}$ to denote the continuous map that injects some $n$-ary product of $D_{\infty}$
%% corresponding to the arity of constructor $K$ into the sum $V_{\infty}$.
%% We use notation $\inj{->}$ to denote the continuous injection of $(D_{\infty} =>_c D_{\infty})$ into $V_{\infty}$ and finally,
%% $\inj{bad}$ for the unit injection into $V_{\infty}$. We use ordinary application notation
%% $d(d')$ \spj{instead of $app(d,d')$?} \dv{No, instead of the @apply@ combinator of the cpo $D_\infty =>_c D_\infty$, which is different
%% than $\dapp$ which has a completely different signature than @apply@ (its arguments are both elements of $D_\infty$)} when $d : D_\infty =>_c D_\infty$ and
%% $d'$ is an element of $D_\infty$. \spj{Why inconsistent notation? Why not say $d':D_{\infty}$?}
%% Indeed application, partial application, and currying are all definable in cpos of continuous functions,
%% so we will be using $\lambda$-calculus notation for our domain theory, as above.


%% \paragraph{Denotational semantics of expressions and programs}
%% =======
\subsection{Contracts}


\begin{figure}
\[\begin{array}{lcl}
e \in \{ x \mid e_p\} & <=> &  e \not\Downarrow \text{ or } e_p[e/x] \not\Downarrow \text{ or} \\
                        &     &  e_p[e/x] \Downarrow True \\
e \in (x{:}\Ct_1) -> \Ct_2 & <=> &
                        \forall e' \in \Ct_1.\; (e\;e') \in \Ct_2[e'/x] \\
e \in \Ct_1 \& \Ct_2 & <=> & e \in \Ct_1 \text{ and } e \in \Ct_2 \\
e \in \CF            & <=> & \forall {\cal C}. BAD \not\in {\cal C} \Rightarrow e \not\Downarrow BAD
\end{array}
\]
\caption{Operational semantics of contracts} \label{f:contract-spec-op}
\end{figure}

We now turn our attention to contracts. The syntax of contracts
is given in Figure~\ref{fig:syntax} and includes base contracts
$\{ x \mid e \}$, arrow contracts $(x : \Ct_1) -> \Ct_2$, conjunctions
$\Ct_1 \& \Ct_2$ and crash-freedom $\CF$. Previous work~\cite{xu+:contracts}
includes other constructs as well, but the constructs we give here are enough to verify
many programs and exhibit all the interesting theoretical and practical problems.

We write $e \in C$ to mean ``the expression $e$ satisfies the contract $C$'', and similarly
for functions $f$.
Figure~\ref{f:contract-spec-op} says what it means to say $e \in C$,
based closely on earlier work \cite{xu+:contracts}.  The specification is
simple, declarative, and intended to be comprehensible to programmers.
For example, an expression $e$ satisfies the function 
contract $\Ct_1 \rightarrow \Ct_2$ if and 
only if $(e\; e')$ satisfies $\Ct_2$ whenever $e'$ satisfies $\Ct_1$.

Crucially, base contracts $\{x|e\}$ allow arbitrary $\cal L$
expressions $e$ (in our implementation, arbitrary Haskell expressions),
rather than being restricted to some well behaved meta-language.  This
is great for the programmer because the language and its library
functions is familiar, but it poses a challenge for verification
because these expressions in contracts may themselves diverge or
crash.


% -----------------------------------------------------------------
\section{Translating $\cal L$ to first-order logic} \label{ssect:denot-fol}

Our goal is to answer the question ``does expression $e$ satisfy
contract $C$?''.  Our plan is to translate both the expression and the
contract into first-order logic (FOL), and get a standard FOL prover
to do the heavy lifting.
In this section we formalise our new translation, and describe how we use it to
verify contracts.

\subsection{The FOL language}

\begin{figure}
\[\begin{array}{c}
\begin{array}{lrll}
\multicolumn{3}{l}{\text{Terms}} \\
  s,t & ::=  & x                          & \text{Variables} \\
      & \mid & f(\ol{t})                  & \text{Function applications} \\
      & \mid & K(\ol{t})                  & \text{Constructor applications} \\
      & \mid & \sel{K}{i}(t)              & \text{Constructor selectors} \\
      & \mid & f_{ptr} \mid app(t,s)       & \text{Pointers and application} \\
      & \mid & \unr \mid \bad             & \text{Unreachable, bad} \\ \\
\multicolumn{3}{l}{\text{Formulae}} \\
 \phi & ::=  & \lcf{t}    & \text{Crash-freedom} \\
%%      & \mid & \lncf{t}   & \text{Can provably cause crash} \\
      & \mid & t_1 = t_2  & \text{Equality} \\
      & \mid & \phi \land \phi \mid \phi \lor \phi \mid \neg \phi \\
      & \mid & \forall x @.@ \phi \mid \exists x @.@ \phi \\ \\
\end{array}
\\
\multicolumn{1}{l}{\text{Abbreviations}} \\
\begin{array}{rcl}
app(t,\oln{s}{n} & = & (\ldots(app(t,s_1),\ldots s_n)\ldots) \\
\phi_1 \Rightarrow \phi_2 & = & \neg \phi_1 \lor \phi_2
\end{array}
\end{array}\]
\caption{Syntax of FOL}\label{fig:fol-image}
\end{figure}

We begin with the syntax of the FOL language, which is given in
Figure~\ref{fig:fol-image}. There are two syntactic forms,
\emph{terms} and \emph{formulae}. Terms include function applications
$f(\ol{t})$, constructor applications $K(\ol{t})$, variables. They
also include, for each data constructor $K^\ar$ in the signature
$\Sigma$ with arity $\ar$ a set of {\em selector functions}
$\sel{K}{i}(t)$ for $i \in 1 \ldots \ar$.  The terms $app(t,s)$ and
$f_{ptr}$ concern the higher-order aspects of $\cal L$, which we
discuss in (see Section~\ref{s:hof}).  Finally we introduce two new
syntactic constructs $\unr$ and $\bad$. As an abbreviation we often use
$app(t,\ol{s})$ for the sequence of applications to each $s_i$, as
Figure~\ref{fig:fol-image} shows.

The formulae of Figure~\ref{fig:fol-image} is just first-order logic
with equality, plus a predicate $\lcf{t}$ for crash-freedom, which
we discuss in Section~\ref{s:cf}.

\subsection{Translation of expressions to FOL}

% ---------------------------------------------------
\begin{figure}\small
\setlength{\arraycolsep}{2pt}
\[\begin{array}{c}
\ruleform{\etrans{\Sigma}{\Gamma}{e} = \formula{t} } \\ \\
\begin{array}{rcl}
\etrans{\Sigma}{\Gamma}{x} & = & \formula{x} \\
\etrans{\Sigma}{\Gamma}{f[\ol{\tau}]} & = & \formula{f_{ptr}} \\
\etrans{\Sigma}{\Gamma}{K[\ol{\tau}](\ol{e})} & = & \formula{K(\ol{\etrans{e}{\Gamma}{t}})} \\
\etrans{\Sigma}{\Gamma}{e_1\;e_2} & = & \formula{app(\etrans{\Sigma}{\Gamma}{t_1},
                                                     \etrans{\Sigma}{\Gamma}{t_1})} \\
\etrans{\Sigma}{\Gamma}{@BAD@} & = & \formula{\bad}
\end{array}
\\ \\
\ruleform{\utrans{\Sigma}{u}{s} = \formula{\phi}} \\ \\
\begin{array}{rcl}
\utrans{\Sigma}{e}{s}
  & = & \formula{(s = \etrans{\Sigma}{\Gamma}{e})} \\
\multicolumn{3}{l}{\utrans{\Sigma}
    {@case@\;e\;@of@\;\ol{K\;\ol{y} -> e'}}{s}} \\
\multicolumn{3}{l}{
\quad
  \begin{array}[t]{rl}
    = & \formula{(t = \bad => s = bad)} \\
    \land & \formula{(\forall \ol{y} @.@ t = K_1(\ol{y}) => s = \etrans{\Sigma}{\Gamma}{e'_1})\;\land \ldots}  \\
    \land & \formula{(t{\neq}\bad\;\land\;
                 t{\neq}K_1(\oln{{\sel{K_1}{i}}(t)}{})\;\land\;\ldots) => s{=}\unr} \\
    \mbox{where} & t  =  \etrans{\Sigma}{\Gamma}{e}
 \end{array}
}
\end{array}
\\ \\
\ruleform{\dtrans{\Sigma}{d} = \formula{\phi}} \\ \\
\begin{array}{rcl}
  \dtrans{\Sigma}{f \;\ol{a}\;\ol{(x{:}\tau)} = u}
    & =     & \formula{\forall \ol{x} @.@ \utrans{\sigma}{u}{f(\ol{x})}} \\
    & \land & \formula{\forall \ol{x} @.@ f(\ol{x}) = app(f_{ptr},\xs)} \\
\end{array}
\\ \\
\ruleform{\ptrans{\Sigma}{P} = \formula{\phi} } \quad
\ptrans{\Sigma}{\ol{d}} = \bigwedge \ol{\dtrans{\Sigma}{d}}
\\ \\
\ruleform{\ctrans{\Sigma}{\Gamma}{e \in \Ct} = \formula{\phi}} \\ \\
\begin{array}{rcl}
\ctrans{\Sigma}{\Gamma}{e \in \{(x{:}\tau) \mid e' \}}
  & = & \formula{t{=}\unr} \\
  & \lor & \formula{t'[t/x]{=}\unr} \\
  & \lor & \formula{t'[t/x]{=}\True} \\
  & \mbox{where} &
    \begin{array}[t]{rcl}
      t  & = & \etrans{\Sigma}{\Gamma}{e} \\
      t' & = & \etrans{\Sigma}{\Gamma}{e'}
    \end{array}
\\
\ctrans{\Sigma}{\Gamma}{e \in (x{:}\Ct_1) -> \Ct_2}
  & = & \formula{\forall x @.@ \ctrans{\Sigma}{\Gamma,x}{x \in \Ct_1}
                          \Rightarrow \ctrans{\Sigma}{\Gamma,x}{e\;x \in \Ct_2}}
\\
\ctrans{\Sigma}{\Gamma}{e \in \Ct_1 \& \Ct_2}
   & = & \formula{ \ctrans{\Sigma}{\Gamma}{e \in \Ct_1} /\ \ctrans{\Sigma}{\Gamma}{e \in \Ct_2}}
\\
\ctrans{\Sigma}{\Gamma}{e \in \CF} & = & \formula{\lcf{\etrans{\Sigma}{\Gamma}{e}}}
\end{array}
\end{array}\]
\caption{Translation of programs and contracts to logic}
   \label{fig:etrans}\label{fig:contracts-minless}
\end{figure}
% ---------------------------------------------------
\begin{figure}\small
\setlength{\arraycolsep}{1pt}
\[\begin{array}{c}
\begin{array}{ll}
\multicolumn{2}{l}{\text{Axioms for $bad$ and $unr$}} \\
 \textsc{AxAppBad}  & \formula{\forall x @.@ app(\bad,x){=}\bad}  \\
 \textsc{AxAppUnr}  & \formula{\forall x @.@ app(\unr,x){=}\unr}    \\
 \textsc{AxDisjBU} & \formula{\bad \neq \unr}  \\
\\
\multicolumn{2}{l}{\mbox{Axioms for data constructors}} \\
 \textsc{AxDisjC} & \formula{\forall \oln{x}{n}\oln{y}{m} @.@ K(\ol{x}) \neq J(\ol{y})} \\
                  & \text{ for every } (K{:}\forall\as @.@ \oln{\tau}{n} -> T\;\as) \in \Sigma \\
                  & \text{ and } (J{:}\forall\as @.@ \oln{\tau}{m} -> S\;\as) \in \Sigma \\
 \textsc{AxDisjCBU} & \formula{(\forall \oln{x}{n} @.@ K(\ol{x}) \neq \unr \; \land \; K(\ol{x}) \neq \bad)} \\
                  & \text{ for every } (K{:}\forall\as @.@ \oln{\tau}{n} -> T\;\as) \in \Sigma \\
 \textsc{AxInj}   & \formula{\forall \oln{y}{n} @.@ \sel{K}{i}(K(\ys)) = y_i} \\
                  & \text{for every } K^\ar \in \Sigma \text{ and } i \in 1..n \\
\\
\multicolumn{2}{l}{\mbox{Axioms for crash-freedom}} \\
 \textsc{AxCfC}  & \formula{\forall \oln{x}{n} @.@ \lcf{K(\ol{x})} <=> \bigwedge\lcf{\ol{x}}} \\
                 & \text{ for every } (K{:}\forall\as @.@ \oln{\tau}{n} -> T\;\as) \in \Sigma \\
 \textsc{AxCfBU} & \formula{\lcf{\unr} /\ \lncf{\bad}} \\
\end{array}
\end{array}\]
\caption{Axioms of the FOL constants}\label{fig:prelude} \label{fig:data-cons}
\end{figure}

% ---------------------------------------------------
What exactly does it mean to translate an expression to first-order logic?
We are primarily interested in reasoning about equality, so we might
hope for this (informal) guiding principle:
$$
e_1 =_S e_2 \;<=>\; \etrans{}{}{e_1} =_F \etrans{}{}{e_2}
$$
where $=_S$ means semantic equality, $=_F$ means equality in first-order logic,
and $\etrans{}{}{e}$ is the translation of $e$ to a FOL term. That is, we can
reason about the equality of Haskell terms by translating them into FOL using
a FOL theorem prover. \spj{Dimitrios, can you forward-ref a theorem that states this?}

The translation of programs, definitions, and expressions to FOL
is given in Figure~\ref{fig:etrans}.
The function $\ptrans{}{P}$ translates a program to a conjunction of formulae,
one for each definition $d$, while $\dtrans{}{d}$ in turn translates
a definition $d$.
The first formula in $\dtrans{}{}$'s right-hand side invokes the translation
$\utrans{}{u}{f(\ol{x})}$ for the right hand side $u$, passing in the term $f(\ol{x})$,
and quantifying over the $\ol{x}$.  We will deal with the second formula shortly.


Ignoring @case@ for now (which we discuss in Section~\ref{s:case-fol}),
the formula $\utrans{}{e}{f(\ol{x})}$
simply asserts the equality $f(\ol{x}) = \etrans{}{}{e}$.
That is, we use a new constant $f$ in the logic for each function definition in the
program, and assert that any application of $f$ is equal to (the logical translation of)
$f$'s right hand side. Notice that we erase type arguments in the translation
since they do not affect the semantics.

Lastly $\etrans{}{}{e}$ deals with expressions.  We will deal with
functions and application next
(Section~\ref{s:hof}), but the other equations for $\etrans{}{}{e}$
are straightforward.  Notice that $\etrans{}{}{@BAD@} = bad$, and recall
that @BAD@ is the $\cal L$-term used for an inexhaustive @case@ or a call
to @error@.  It follows from our guiding principle
that for any $e$, if
$$ \etrans{}{}{e} =_F bad $$
in the logic, then the source program $e$
must be semantically equivalent to @BAD@, meaning that it definitely
crashes by calling @error@, or having an inexhaustive @case@.

\spj{We ought to say a bit more about $unr$ somewhere}

\subsection{Translating higher-order functions} \label{s:hof}

If $\cal L$ was
a first-order language the translation of function calls would be easy:
$$
\etrans{\Sigma}{\Gamma}{f[\ol{\tau}]\;\ol{e}} = \formula{f(\ol{\etrans{}{}{e}})} \\
$$
At first it might be surprising that we can also translate a \emph{higher-order} language
$\cal L$ into first order logic, but in fact it is easy to do so, as
Figure~\ref{fig:etrans} shows.  We introduce into the logic
(a) a single new constant $app$, standing
for application, and (b) a nullary constant $f_{ptr}$ for each function $f$
(see Figure~\ref{fig:fol-image}).
Then, the equations for $\etrans{}{}{e}$ translate application in $\cal L$ to
a use of $app$ in FOL, and any mention of function $f$ in $\cal L$ to a use
of $f_{ptr}$ in the logic.  For example:
$$
\etrans{}{}{@map f xs@} = app( app( @map@_{ptr}, @f@_{ptr}), @xs@)
$$
assuming that @map@ and @f@ are top-level functions in the $\cal L$-program, and
@xs@ is a local variable.  Once enough $app$ applications stack up, so that
$@map@_{ptr}$ is applied to two arguments, can invoke @map@ directly in the logic,
an idea we expression with the following axiom:
$$
\forall x y.\;app(app(@map@_{ptr}, x), y) = @map@(x,y)
$$
These axioms, one for each function $f$, are generated by the second
clause of the rules for $\dtrans{}{d}$ in Figure~\ref{fig:etrans}.
(The notation $app(f,\ol{x})$ is defined in Figure~\ref{fig:fol-image}.)
You can think of $@map@_{ptr}$ as a ``pointer to'', or ``name of'' of, @map@.
The $app$ axiom for @map@ translates a saturated use of @map@'s pointer into
a call of @map@ itself.

\subsection{Data types and {\tt case} expressions} \label{s:case-fol}

The second equation for $\utrans{}{u}{s}$ in Figure~\ref{fig:etrans} deals with
@case@ expressions, by generating a conjunction of formulae, as follows:
\begin{itemize}
\item If the scrutinee $t$ is $bad$ (meaning that evaluating it invokes @BAD@) then
the result $s$ of the @case@ expression is also $bad$.  That is, @case@ is strict in
its scrutinee.
\item If the scrutinee is an application of one of the constructors $K_i$ mentioned
in one of the @case@ alternatives, then the result $s$ is equal to the corresponding
right-hand side, $e'_i$, after quantifying the variables $\ol{y}$ bound by the @case@ alternative.
\item Otherwise the result is $unr$.
The bit before the implication $\Rightarrow$ is just the
negation of the previous preconditions; the formula
  $t{\neq}K_1(\oln{{\sel{K_1}{i}}(t)}{})$
is the clumsy FOL way to say ``$t$ is not built with constructor $K_1$.
\end{itemize}
Why do we need the last clause? Consider the function @not@:
\begin{code}
  not :: Bool -> Bool
  not True = False
  not False = True
\end{code}
Suppose we want to prove that $@not@ \in @CF@ \rightarrow @CF@$.
If we lack the last clause above, the FOL prover would have
no way to dismiss the possibility that $@not@(@3@) =_F bad$, and thus
would fail to prove the theorem.  However, the type system guarantees
that @not 3@ will never show up, and the final clause is the way we express that
fact to the FOL prover.

Of course, we also need to axiomatise the behaviour of data constructors and
selectors, which is done in Figure~\ref{fig:data-cons}:
\begin{itemize}
\item \textsc{AxDisjCBU} explains that a term headed by a data constructor cannot
also be $bad$ or $unr$.
\item \textsc{AxInj} explains how selectors $\sel{K}{i}$ work.
\item \textsc{AxDisjC} tells the prover that all data constructors are pairwise disjoint.
There are a quadratic number of such axioms, which presents a scaling problem.
For this reason FOL provers sometimes offer a built-in notion of data constructors,
so this is not a problem in practice, but we ignore this pragmatic issue here.
\end{itemize}

\subsection{Translation of contracts to FOL} \label{s:contracts-fol}

Now that we know how to translate \emph{programs} to first order
logic, we turn our attention to translating \emph{contracts}.  We do
not translate a contract \emph{per se}; rather we translate the claim
$e \in C$.  Once we have translated $e \in \Ct$ to a first-order logic
formula, we can ask the prover to prove it; and, if successful, we can
claim that indeed $e$ does satisfy $C$.  Of course that needs proof,
which we address in Section~\ref{xxx}.

Figure~\ref{fig:contracts-minless} presents the translation
$\ctrans{}{\Gamma}{e \in \Ct}$; there are four equations corresponding
to the syntax of contracts in Figure~\ref{fig:syntax}.
The last three cases are delightfully simple and direct.  Conjunction of contracts
turns into conjunction in the logic; a dependent function contract turns
into universal quantification and implication; and the claim that $e$ is
crash-free turns into a use of the special term $\lcf{t}$ in the logic.
We discuss crash-freedom in Section~\ref{s:cf-fol}.

The first equation, for predicate contracts $e \in \{(x{:}\tau) \mid e' \}$,
is sightly more complicated.
The first clause $t=unr$ says that the contract holds if $e$ diverges \spj{Is this right?}.
The second and third say that the contract holds if $e'$ diverges or is semantically
equal to @True@.  The choices embodied in this rule were discussed at length
in earlier work \cite{xu+:contracts} and we do not rehearse it here.

\subsection{Crash-freedom} \label{s:cf-fol}

The claim $e \in @CF@$, pronounced ``$e$ is crash-free'', means that $e$ cannot
crash \emph{regardless of context}.  So, for example @(BAD, True)@ is not crash-free
because it can crash if evaluated in the context @fst (BAD, True)@.  Of course,
the context itself should not be the source of the crash; for example @(True,False)@ is
crash-free even though @BAD (True,False)@ will crash.

We use the FOL term $\lcf{t}$ to assert that $t$ is crash-free. The axioms for $\lcf{}$
are given in Figure~\ref{fig:prelude}.  \textsc{AxCfC} says that a data constructor application
is crash-free if and only iff its arguments are crash-free.  \textsc{AxCfBU} says that
$unr$ is crash-free, and that $bad$ is not.  That turns out to be all that we need.

\subsection{Summary}

That completes our formally-described --- but only informally-justified --- translation
from a $\cal L$ program and a set of contract claims, into first-order logic.
To a first approximation, we can now turn the whole formula over to a FOL prover,
and hope that it proves the assertion.

\spj{What else should we say here.  Something about recursion?  About multiple functions?}



\section{Contracts and their denotational semantics}\label{sect:contracts}
  To the extend that in the end we are only interested in base contracts, giving a 
denotational semantics of full-higher-order contracts is not really interesting 
but we do this anyway. For a given denotation $d$, we define the 
predicate $\interp{\Ct}{\dbrace{P}^\infty}{\rho}(d)$ by recursion on the structure 
of the contract $\Ct$, such that:

\[\begin{array}{l}
    \interp{\{x \mid e\}}{\dbrace{P}^\infty}{\rho}(d) \text{ iff } \\
        \quad \unroll(d) = \bot \text{ or } 
        \unroll(\interp{e}{\dbrace{P}^\infty}{\rho,x|->d}) = \bot\;\text{ or } \\
        \quad \unroll(\interp{e}{\dbrace{P}^\infty}{\rho,x|->d}) = \ret(\inj{\mathop{True}}(1)) \\ \\
    \interp{\dbrace{(x{:}\Ct_1) -> \Ct_2}{P}^\infty}{\rho}(d) \text{ iff } \\
        \quad \text{for all } d_x \in D_\infty \\ 
        \quad\quad \text{if }
                     \interp{\Ct_1}{\dbrace{P}^\infty}{\rho}(d_x)\text{ then }
                     \interp{\Ct_2}{\dbrace{P}^\infty}{\rho,x|->d_x}(\dapp(d,d_x)) \\ \\ 
    \interp{\CF}{\dbrace{P}^\infty}{\rho}(d) \text{ iff }  d \in \Fcf^{\infty} \\  \\ 
    \interp{\Ct_1 \& \Ct_2}{\dbrace{P}^\infty}{\rho}(d) \text{ iff } 
       \interp{\Ct_1}{\dbrace{P}^\infty}{\rho}(d) \text{ and } 
       \interp{\Ct_2}{\dbrace{P}^\infty}{\rho}(d)
\end{array}\]

\subsection{Contracts in first-order logic}\label{sect:contracts-fol}

\begin{figure}\small
\[\begin{array}{c} 
\ruleform{\ctrans{\Sigma}{\Gamma}{e \in \Ct} = \formula{\phi}} \\ \\ 
\prooftree
  \begin{array}{c}
   \etrans{\Sigma}{\Gamma}{e} = \formula{t} \quad
   \etrans{\Sigma}{\Gamma,x}{e'} = \formula{t'}
  \end{array}
  ------------------------------------------{CTransBase}
  \begin{array}{l}
   \ctrans{\Sigma}{\Gamma}{e \in \{(x{:}\tau) \mid e' \}} = \\
  %% \Sigma;\Gamma |- e \in \{(x{:}\tau \mid e' \}
  \;\;\formula{(t = \unr) \lor (t'[t/x] = \unr) \lor (t'[t/x] = \True)}
  \end{array}
  ~~~~~ 
  \begin{array}{c}
  \ctrans{\Sigma}{\Gamma,x}{x \in \Ct_1} {=} \formula{\phi_1} \quad
  \ctrans{\Sigma}{\Gamma,x}{e\;x \in \Ct_2} {=} \formula{\phi_2}
  \end{array} 
  ------------------------------------------{CTransArr}
  \begin{array}{l} 
  \ctrans{\Sigma}{\Gamma}{e \in (x{:}\Ct_1) -> \Ct_2} = 
  \formula{\forall x @.@ \neg \phi_1 \lor \phi_2} 
  \end{array}
  ~~~~~
  \begin{array}{c}
  \ctrans{\Sigma}{\Gamma}{e \in \Ct_1} = \formula{ \phi_1} \quad
  \ctrans{\Sigma}{\Gamma}{e \in \Ct_2} = \formula{ \phi_2}
  \end{array}
  ------------------------------------------{CTransConj}
  \ctrans{\Sigma}{\Gamma}{e \in \Ct_1 \& \Ct_2} = \formula{ \phi_1 /\ \phi_2}
  ~~~~~
  \etrans{\Sigma}{\Gamma}{e} =  \formula{t}
  -------------------------------------------{CTransCf}
  \ctrans{\Sigma}{\Gamma}{e \in \CF} = \formula{\lcf{t}}
 \endprooftree 
%% \\ \\ 
%% \ruleform{\Sigma;\Gamma |- e \notin \Ct \elab{ \phi} } \\ \\
%% \prooftree
%%   \begin{array}{c}
%%    \Sigma;\Gamma |- e : \tau \elab{ t}  \quad
%%    \Sigma;\Gamma,(x{:}\tau) |- e' : \Bool \elab{ t'}
%%   \end{array}
%%   ------------------------------------------{CNTransBase}
%%   \begin{array}{l}
%%   \Sigma;\Gamma |- e \notin \{(x{:}\tau) \mid e' \} 
%%   \elab{(t'[t/x] = \bad) \lor (t'[t/x] = \False)}
%%   \end{array}
%%   ~~~~~ 
%%   \Sigma;\cdot |- \Ct_1 : \tau
%%   ------------------------------------------{CNTransArr}
%%   \begin{array}{l} 
%%   \Sigma;\Gamma |- e \notin (x{:}\Ct_1) -> \Ct_2 
%%   \elab{\exists x @.@ (\Sigma;\Gamma,(x{:}\tau) |- x \in \Ct_1) /\ (\Sigma;\Gamma,(x{:}\tau) |- e\;x \notin \Ct_2)}
%%   \end{array}
%%   ~~~~~
%%   \begin{array}{c}
%%   \Sigma;\Gamma |- e \notin \Ct_1 \elab{ \phi_1} \quad
%%   \Sigma;\Gamma |- e \notin \Ct_2 \elab{ \phi_2}
%%   \end{array}
%%   ------------------------------------------{CNTransConj}
%%   \Sigma;\Gamma |- e \notin \Ct_1 \& \Ct_2 \elab{ \phi_1 \lor \phi_2}
%%   ~~~~
%%   \Sigma;\Gamma |- e : \tau \elab{ t}
%%   -------------------------------------------{CNTransCf}
%%   \Sigma;\Gamma |- e \notin \CF \elab{ \lncf{t}}
%%  \endprooftree
\end{array}\]
\caption{Baseline contract elaboration}\label{fig:typing}
\end{figure}


In this section we will attempt to ignore the higher-order case and just talk about 
base contracts. Let us use:

The following are true: 
\begin{lemma}[Base contract adequacy]\label{lem:base-contract-adequacy}
Assume that $\Sigma |- P$ and $fv(e) \subseteq dom(P)$, i.e. $e$ is closed.
If $\langle D_\infty,{\cal I}\rangle \models \ctrans{\Sigma}{\Delta}{e \in \CF}$ then $P |- e \in \CF$. If $\langle D_\infty,{\cal I}\rangle \models \ctrans{\Sigma}{\Delta}{e \in \{ x \mid e' \}}$ then $P |- e \in \{x \mid e' \}$.
\end{lemma}
{\bf DV: Generalize this to a notion of base contracts that includes conjuctions.}

In fact the above two statements hold if we extend the interpretation 
of crash-freedom in the model to contain elements from the function 
space as well. 

Because of the full-abstraction problems we have observed above it 
is not possible to state similar statements for arrow contracts. 



\subsection{Soundness of contract checking}\label{ssect:soundness}


\subsubsection{Invocation of a theorem prover}\label{sect:infocation}

Given a program $P$ with signature $\Sigma$, that is $\Sigma |- P$, we may define the theory
${\cal T}$ as follows:
     \[ \Th{\Sigma}{P}\;\land\;\Th{\Sigma}{P}^{\lcfZ}\;\land\;\dtrans{\Sigma}{P} \]
we know that $\langle D_\infty,{\cal I}\rangle \models {\cal T}$ from the previous sections. 
Assume below that $f$ is a function such that $f \in dom(P)$ and $fv(\Ct) \subseteq dom(P)$.

\begin{theorem}[Soundness]\label{thm:prover-soundess}
If ${\cal T}\;\land\;\neg(\ctrans{\Sigma}{P}{f \in \Ct})$ is 
unsatisfiable then $\langle D_\infty,{\cal I}\rangle \models \ctrans{\Sigma}{P}{f \in \Ct}$.
\end{theorem}
\begin{proof}
If there is no model for this formula (i.e. the theorem prover returns ``unsatisfiable'') then
its negation must be valid (true in all models), that 
is ${\cal T} -> \ctrans{\Sigma}{P}{f \in \Ct}$ is valid. By completeness
of first-order logic ${\cal T} |- \ctrans{\Sigma}{P}{f \in \Ct}$. This means in 
turn that all models of ${\cal T}$ validate $\ctrans{\Sigma}{P}{f \in \Ct}$. In particular 
for the denotational model we have that $\langle D_\infty,{\cal I}\rangle \models {\cal T}$ 
and hence $\langle D_\infty,{\cal I}\rangle \models \ctrans{\Sigma}{P}{f \in \Ct}$.
\end{proof}

\subsubsection{End-goal and incremental verification}\label{sect:incremental}

Assume that we are given a program $P$ with a function $f \in dom(P)$, for which we have 
already showed that $\langle D_\infty,{\cal I}\rangle \models \ctrans{\Sigma}{P}{f \in \Ct_f}$. 
Suppose next that we are presented with a ``next'' goal, to prove that 
$\langle D_\infty,{\cal I}\rangle \models \ctrans{\Sigma}{P}{h \in \Ct}$. 
We may consider the following three variations of how to do this:

\begin{itemize}
  \item Simply ask for the unsatisfiability of: 
    \[  \Th{\Sigma}{P}\;\land\;
        \Th{\Sigma}{P}^{\lcfZ}\;\land\;\dtrans{\Sigma}{P}\;\land\;\neg\ctrans{\Sigma}{P}{h \in \Ct_h} \] 
        The soundness of this query follows directly from Theorem~\ref{thm:prover-soundess} above.

  \item Ask for the unsatisfiability of:
    \[  \Th{\Sigma}{P}\;\land\;
        \Th{\Sigma}{P}^{\lcfZ}\;\land\;\dtrans{\Sigma}{P}\;\land\;\ctrans{\Sigma}{P}{f \in \Ct_f}\;\land\;\neg\ctrans{\Sigma}{P}{h \in \Ct_h}     \] 
        This query adds the {\em already proven} contract for $f$ in the theory. If this formula
        is unsatisfiable, then its negation is valid, and we know that the denotational model is 
        a model of the theory {\em and} of $\ctrans{\Sigma}{P}{f \in \Ct_f}$ and hence it must also
        be a model of $\ctrans{\Sigma}{P}{h \in \Ct_h}$. 
  \item Ask for the unsatisfiability of:
    \[  \Th{\Sigma}{P}\;\land\;
        \Th{\Sigma}{P}^{\lcfZ}\;\land\;\dtrans{\Sigma}{P \setminus f}\;\land\;\ctrans{\Sigma}{P}{f \in \Ct_f}\;\land\;
        \neg\ctrans{\Sigma}{P}{h \in \Ct_h}     \] 
        This query removes the axioms associated with the definition of $f$ since we may only have 
        its signature and contract available. Via a similar reasoning as before, such an invocation 
        is sound.
\end{itemize}

Our final goal is going to show that a program does not crash, that
is the final contract will be of the form $e \in \Ct$ where $\Ct$ is
going to be some {\em base contract}. Note that by base contract adequacy
(Lemma~\ref{lem:base-contract-adequacy}) if we manage to show a base contract 
denotationally, then the contract holds in operational terms.



\subsection{Denotational versus operational semantics for contracts}
TODO -- I have just dumpted material here. 

We have the rather obvious theorem below.

\begin{theorem}[Soundness and completeness for denotational semantics]
Assume a program $P$ with signature $\Sigma$, and expression $e$ and contract $\Ct$ 
such that $fv(e) \cup fv(\Ct) \subseteq dom(P)$. Then 
$\langle D_\infty,{\cal I}\rangle \models \ctrans{\Sigma}{P}{e \in \Ct}$ iff
$\interp{\Ct}{\dbrace{P}^{\infty}}{\cdot}(\interp{e}{\dbrace{P}^\infty}{\cdot})$.
\end{theorem}




\subsubsection{Contract satisfaction and crash-freedom}\label{sect:cf}

We would like to define a set of contract-satisfying denotations and also a set of contract-satisfying terms, 
characterized by $P |- e \in \Ct$, such that the following claim becomes true:

\begin{proposition} Assume that $\Sigma |- P$ and $fv(e) \subseteq dom(P)$, i.e. $e$ is closed.
Then: $\langle D_\infty,{\cal I}\rangle \models \ctrans{\Sigma}{\Delta}{e \in \Ct}$ iff $P |- e \in \Ct$.
\end{proposition}

Now there are several problems with coming up with a good definition of $P |- e \in \Ct$, 
which we elaborate in the following sections.

\subsubsection{Problem I: Crash-freedom} 

Ideally we would like to define crash-freedom {\em semantically} using the following 
strict bifunctor on admissible sets $S^{-},S^{+} \subseteq D_{\infty}$.
{\setlength{\arraycolsep}{2pt}
\[\begin{array}{rcl}
   F_{\lcfZ}(S^{-},S^{+}) & = & \{\;d\;\mid\;\unroll(d) \neq \ret(\inj{bad}(1))\;\land\; \\ 
                      &    & \quad \forall \ol{d} @.@ \unroll(d){=}\ret(\inj{K_1^\ar}\langle\oln{d}{\ar}\rangle) ==> \ol{d} \in S^{+} \} \\ 
                   & \cup & \ldots \\ 
                   & \cup & \{\;d\;\mid\;\unroll(d) \neq \ret(\inj{bad}(1))\;\land\; \\ 
                   &      & \quad \forall d_0 @.@ \unroll(d) = \ret(\inj{->}(d_0)) ==> \\ 
                   &      & \quad\quad \forall\;d' \in S^{-} ==> \dapp(d,d') \in S^{+} \}  \\
\end{array}\]}
The $\Fcf$ bifunctor has a negative and positive fixpoint, and by minimal invariance they coincide (one direction 
follows by Tarski-Knaster, the other can be inductively proved using the approximations on ever element of $D_{\infty}$ given
in Lemma~\ref{lem:min-inv-reqs} and the fact that the lub of the chain of $\rho_i$ is the identity and the fact that this 
functor preserves admissibility for the positive sets). Let us call this admissible set $\Fcf^{\infty} \subseteq D_{\infty}$.

We consider this predicate to be the ``ideal crash-freedom'' -- however it is very difficult to give a 1-1 operational
definition. The reason is that the $\Fcf$ functor quantifies in the function case over any $d'$ -- whereas in the operational
semantics it is only reasonable that we quantify over all terms (or over terms that do not contain @BAD@) In the absense of 
full abstraction of the domain (which is plausible, especially if we extend the language with other features) it is unclear 
what a corresponding predicate would look like in terms of operational semantics. 

We then go for a simpler predicate, which only characterizes crash-freedom for first-order terms, 
generate by the following functor on {\em admissible} sets of denotations:
{\setlength{\arraycolsep}{2pt}
\[\begin{array}{rcl}
   G_{\lcfZ}(S^{+}) & = & \{\;d\;\mid\; \unroll(d){=}\ret(\inj{K_1^\ar}\langle\oln{d}{\ar}\rangle) \land \ol{d} \in S^{+} \} \\ 
                  & \cup & \ldots \\ 
                  & \cup & \{\;\bot\;\}
\end{array}\]}
Notice that if $S$ is admissible then so is $G_{\lcfZ}(S)$. 

%% The $G_{\lcfZ}$ functor has a fixpoint and it is an admissible relation, and we will use its 
%% fixpoint $G_{\lcfZ}^\infty$, so now we need to say what $G_{\lcfZ|}^\infty$ means operationally. 
\begin{lemma} The functor $G_{\lcfZ}$ has a unique fixpoint $G_{\lcfZ}^\infty$ on admissible sets. \end{lemma}
\begin{proof} 
The intersection of admissible sets is admissible. Hence we have a complete join semi-lattice (which induces a 
complete lattice), so the monotone functor $G_{\lcfZ}$ does have a smallest and a greatest fixpoint call
it $G_{\lcfZ}^{min}$ and $G_{\lcfZ}^{max}$. Moreover this fixpoint will be an admissible relation. Now it must be 
that $G_{\lcfZ}^{min} \subseteq G_{\lcfZ}^{max}$ so we only show next that
also $G_{\lcfZ}^{max} \subseteq G_{\lcfZ}^{min}$. To do this we will show that:
\[ \forall i. d \in G_{\lcfZ}^{max} ==> \rho_i(d) \in G_{\lcfZ}^{min} \] 
by induction on $i$. For $i = 0$ it follows since $\rho_0(d) = \bot$. Let us assume 
that it holds for $i$, we need to show that $\rho_{i+1}(d) \in G_{\lcfZ}(G_{\lcfZ}^{min})$.
We know however that $d \in G_{\lcfZ}(G_{\lcfZ}^{max}$ and by simply case analysis and appealing
to the induction hypothesis we are done. Finally, by admissibility it must be that
$\sqcup\rho_i(d) \in G_{\lcfZ}^{min}$ and by Lemma~\ref{lem:min-inv-reqs} it
must be that $d \in G_{\lcfZ}^{min}$. This means that the two fixpoints coincide, 
hence there is only a unique fixpoint of $G_{\lcfZ}$, call it $G_{\lcfZ}^\infty$.
\end{proof} 

Now, we would like to define operationally the set of {\em crash-free} terms as a set $\Ecf$ of 
closed terms that satisfies:
{\setlength{\arraycolsep}{2pt}
\[\begin{array}{rcl}
   \Ecf & =    & \{ e \;\mid\; P |- e \Downarrow K[\taus](\ol{e}) /\ \ol{e} \in \Ecf \} \\
        & \cup & \ldots \\
        &      & \{ e \;\mid\; P \not|- e \Downarrow \} 
\end{array}\]}%
We do not know that the set $\Ecf$ exists, so we have to prove it. 
\begin{lemma}
There exists a largest set that satisfies the $\Ecf$ equation above.
\end{lemma} 
\begin{proof}
Define $\Ecf$ to be the set
\[ \{ e\;\mid\; \interp{e}{\dbrace{P}^\infty}{\cdot} \in G_{\lcfZ}^{\infty}\} \]
It is straightforward (by computational adequacy) to show that it satisfies the $\Ecf$ recursive
equation above. For uniqueness, assume any other set $E$ that satisfies the recursive equation
above. We can show that $\interp{E}{\dbrace{P}^\infty}{\cdot}$ is a
fixpoint of $G_{\lcfZ}$ and since there is only one such fixpoint, this is unique. So we have that:
\[\begin{array}{ll}
 e \in E & ==> \\ 
 \interp{e}{\dbrace{P}^\infty}{\cdot} \in \interp{E}{\dbrace{P}^\infty}{\cdot} & ==> \\
 \interp{e}{\dbrace{P}^\infty}{\cdot} \in G_{\lcfZ}^\infty & ==> \\
 e \in \Ecf 
\end{array}\] 
\end{proof}
%% \begin{lemma} 
%% If $e \in E$ and $\interp{e}{\dbrace{P}^\infty}{\cdot} = \interp{e'}{\dbrace{P}^\infty}{\cdot}$ then $e' in E$.
%% \end{lemma}
%% This relies on the fact that 
%% if $\interp{e}{\dbrace{P}^\infty}{\cdot} \in \interp{E}{\dbrace{P}^\infty}{\cdot}$ then $e \in E$. Why is 
%% that? Because the assumption means that 
%% $\interp{e}{\dbrace{P}^\infty}{\cdot} \in \{ d | \exists e' \in E /\ d = \interp{e'}{\dbrace{P}^\infty}{\cdot} \}$
%% and hence this means that there exists some $e' \in E $ such that 
%% $\interp{e}{\dbrace{P}^\infty}{\cdot} = \interp{e'}{\dbrace{P}^\infty}{\cdot}$ 
%% \end{proof} 

Let us extend the interpretation function above $\linterp{\cdot}$ so that: 
\[\begin{array}{rcl}
   \linterp{\lcfZ}  & = & G_{\lcfZ}^{\infty} 
\end{array}\]

\begin{theorem}
If $\Sigma |- P$ then we have that $\langle D_{\infty},{\cal I}\rangle \models \Th{\Sigma}{P}^{\lcfZ}$.
\end{theorem}

Notice that the axiom:
\[  \textsc{AxCfC}  \quad \formula{\forall x y @.@ \lcf{x} /\ \lcf{y} => \lcf{app(x,y)}} \]
is {\em not validated} by this interpretation of crash-freedom we have given. 


\subsubsection{Problem II: the absense of full-abstraction}

Unfortunately higher-orderness bites again. Having defined the set $\Ecf$ we might define formally
the predicate $P |- e \in \Ct$ where $fv(e) \subseteq dom(P)$ and $fv(\Ct) \subseteq dom(P)$ as 
follows:
{\setlength{\arraycolsep}{2pt}
\[\begin{array}{lcl}
    P |- e \in \{ x\;\mid\;e_p\} & <=> & P |- e \not\Downarrow \text{ or } P |- e_p[e/x] \not\Downarrow \text{ or} \\ 
                                 &     & P |- e_p[e/x] \Downarrow True \\
    P |- e \in (x{:}\Ct_1) -> \Ct_2 & <=> & 
                                 \text{for all } P' e' \text{ s.t. } fv(e') \subseteq dom(P{\uplus}P')  \\ 
                                   &   &  \text{it is } P\uplus P' |- e\;e' \in \Ct_2[e'/x] \\
    P |- e \in \Ct_1 \& \Ct_2 & <=> & P |- e \in \Ct_1 \text{ and } P |- e \in \Ct_2 \\
    P |- e \in \CF            & <=> & e \in \Ecf 
\end{array}\]}

Note we made the definition above well-scoped but not necessarily well-typed; let's ignore that for now (making everything
well-typed includes extra difficulties in the proof but hopefully not surmountable).

The interesting case is the case for arrow contracts, where we have extended the set of definitions $P$ with more 
definitions $P'$ -- that is to allow for tests $e'$ which can have arbitrary computational power, and not only those
that can be constructed in the current environment. That is expected the way we have set up things, so let us examine
what happens when we try to prove the proposition below:

\begin{proposition} Assume that $\Sigma |- P$ and $fv(e) \subseteq dom(P)$, i.e. $e$ is closed.
Then: $\langle D_\infty,{\cal I}\rangle \models \ctrans{\Sigma}{\Delta}{e \in \Ct}$ iff $P |- e \in \Ct$.
\end{proposition}

{\flushleft{\em Failed proof}:}
The base case and the case of $\CF$ follow from computational adequacy so we are good. However
let's try to prove the arrow case and in particular the $(<=)$ direction. 

Let us assume that for all $P'$ and $e'$ such that $fv(e') \subseteq dom(P\uplus P')$ it is the case that
$P |- e\;e' \in \Ct_2[e'/x]$. We need to show that $\langle D_\infty,{\cal I}\rangle$ is a model of the 
formula $\forall x. \ctrans{\Sigma}{x}{x \in \Ct_1} => \ctrans{\Sigma}{x}{e\;x \in \Ct_2}$. Let us fix
a denotation $d \in D_{\infty}$ and let us assume 
that $\langle D_{\infty},{\cal I} \rangle \models \ctrans{\Sigma}{x}{x \in \Ct_1}[d/x]$. However, this does not 
necessarily mean that we can find a closed $e'$ and $P'$, such 
that $\interp{e'}{\dbrace{P{\uplus}P'}^\infty}{\cdot} = d$ to be able to use the assumptions, unless some sort
of full-abstraction property is true. So we are stuck.

Here is a concrete counterexample, based on the lack of full-abstraction due to the {\em parallel or} function. 
Consider the program $P$ below:
\[\begin{array}{lcl}
f_\omega & |-> & f_\omega \\
f & |-> & \lambda (b{:}Bool) @.@ \lambda (h{:}Bool->Bool->Bool) @.@ \\
  &     & \quad @if@\;(h\;True\;b)\;\&\&\;(h\;b\;True)\;\&\& \\ 
  &     & \quad\qquad\qquad not\;(h\;False\;False)\;@then@ \\
  &     & \quad\quad @if@\;(h\;True\;f_\omega)\;\&\&\;(h\;f_\omega\;True)\;@then@\;@BAD@ \\
  &     & \quad\quad @else@\;True \\
  &     & \quad @else@\;True
\end{array}\]
Consider now the candidate contract for $f$ below: 
\[ \CF -> (\CF -> \CF -> \CF) -> \CF \]
Operationally we may assume a crash-free boolean as well as a function $h$ which is 
$\CF -> \CF -> \CF$. The first conditional ensures that the function behaves like an ``or'' function or 
diverges. However if we pass the first conditional, 
the second conditional will always diverge and hence the contract will be satisfied. 

However, denotationally it is possible to have a {\em monotone} function $por$ defined as follows:
\[\begin{array}{lcl}
  por\;\bot\;\bot & = & \bot \\ 
  por\;\bot\;True & = & True \\
  por\;True\;\bot & = & True \\ 
  por\;False\;False & = & False
\end{array}\] 
with the rest of the equations (for @BAD@ arguments) induced by monotonicity and whatever boolean value 
we like when both arguments are @BAD@. 

Now, this is denotationally a $\CF -> \CF -> \CF$ function, and it will pass the first conditional, but it will
also pass the second conditional, yielding @BAD@. Hence denotationally the contract for $f$ {\em does not hold}.

So we have a concrete case where the $<=$ direction fails. Because of contra-variance of arrow contracts, it is 
likely that the $=>$ direction is false as well. 


%% Now it may be the case that for all denotations that semantically satisfy a contract, these denotations {\em are} 
%% realizable by a term $e'$ and a context $P'$ but it is not entirely clear how to prove this (or if this is a good
%% idea). I am not sure if this is true either.
%% The other idea out of this situation is to compile the arrow contract differently by not quantifying over all 
%% denotations but rather some kind of {\em definable} denotations -- but I do not know how exactly to do this.


\paragraph{A way out of this?}
Well, if we restrict our higher-order tests to those that can be constructed from our signature then 
we may define the following:

{\setlength{\arraycolsep}{2pt}
\[\begin{array}{lcl}
    P |- e \in \{ x\;\mid\;e_p\} & <=> & P |- e \not\Downarrow \text{ or } P |- e_p[e/x] \not\Downarrow \text{ or} \\ 
                                 &     & P |- e_p[e/x] \Downarrow True \\
    P |- e \in (x{:}\Ct_1) -> \Ct_2 & <=> & 
                                 \text{for all } e' \text{ s.t. } fv(e') \subseteq dom(P)  \\ 
                                   &   &  \text{it is } P |- e\;e' \in \Ct_2[e'/x] \\
    P |- e \in \Ct_1 \& \Ct_2 & <=> & P |- e \in \Ct_1 \text{ and } P |- e \in \Ct_2 \\
    P |- e \in \CF            & <=> & e \in \Ecf 
\end{array}\]}
Notice that the difference with the previous version of $P |- e \in \Ct$ is that we {\em do not} extend the 
definitions $P'$ so we don't get the full power of higher-order tests. We show that {\em in the current signature
only} does the program satisfy the contract. 


Why did we do this change? Because denotationally this is not terribly hard to support -- instead of translating 
\[\begin{array}{l}
  \ctrans{\Sigma}{\Gamma}{e \in (x{:}\Ct_1) -> \Ct_2} =  
  \formula{\forall x @.@ \ctrans{\Sigma}{\Gamma,x}{x \in \Ct_1} => \ctrans{\Sigma}{\Gamma,x}{e\;x \in \Ct_2}}
\end{array}\] 
we use the following:
\[\begin{array}{l}
  \ctrans{\Sigma}{\Gamma}{e \in (x{:}\Ct_1) -> \Ct_2} = \\ 
  \qquad\qquad\quad 
\formula{\forall x @.@ \definable{x} \land \ctrans{\Sigma}{\Gamma,x}{x \in \Ct_1} => \ctrans{\Sigma}{\Gamma,x}{e\;x \in \Ct_2}}
\end{array}\] 
where $\definable{x}$ could be axiomatized as containing all terms 
made up of the functions in $P$, applications, and data constructors:

\[\begin{array}{lll} 
 \textsc{DefCons} & \formula{\forall \xs @.@ \definable{K(\xs)} <=> \definable{\xs}} \\
                        & \text{ for every } (K{:}\forall\as @.@ \oln{\tau}{n} -> T\;\as) \in \Sigma \\
 \textsc{DefFun}  & \formula{\definable{f_{ptr}}}  \\
                        & \text{ for every } (f |-> \Lambda\as @.@ \lambda\oln{x{:}\tau}{n} @.@ u) \in P \\
 \textsc{DefApp}  & \formula{\forall x y @.@ \definable{x}\land\definable{y} => \definable{app(x,y)}}
%% \formula{\bad \neq \unr}  \\ 
%%  \textsc{AxDisjB} & \formula{\forall \oln{x}{n}\oln{y}{m} @.@ K(\ol{x}) \neq J(\ol{y})} \\ 
%%                   & \text{ for every } (K{:}\forall\as @.@ \oln{\tau}{n} -> T\;\as) \in \Sigma \\ 
%%                   & \text{ and } (J{:}\forall\as @.@ \oln{\tau}{m} -> S\;\as) \in \Sigma \\
%%  \textsc{AxDisjC} & \formula{(\forall \oln{x}{n} @.@ K(\ol{x}) \neq \unr \land K(\ol{x}) \neq \bad)} \\ 
%%                   & \text{ for every } (K{:}\forall\as @.@ \oln{\tau}{n} -> T\;\as) \in \Sigma \\ \\
%%  \textsc{AxAppA}  & \formula{\forall \oln{x}{n} @.@ f(\ol{x}) = app(f_{ptr},\xs)} \\
%%                   & \text{ for every } (f |-> \Lambda\as @.@ \lambda\oln{x{:}\tau}{n} @.@ u) \in P \\
%%  %% \textsc{AxAppB}  & \formula{\forall \oln{x}{n} @.@ K(\ol{x}) = app(\ldots (app(x_K,x_1),\ldots,x_n)\ldots)} \\
%%  %%                  & \text{ for every } (K{:}\forall\as @.@ \oln{\tau}{n} -> T\;\as) \in \Sigma \\
%%  \textsc{AxAppC}  & \formula{\forall x, app(\bad,x) = \bad \; /\ \; app(\unr,x) = \unr}    \\ \\
%%  %% Not needed: we can always extend partial constructor applications to fully saturated and use AxAppC and AxDisjC
%%  %% \textsc{AxPartA} & \formula{\forall \oln{x}{n} @.@ app(\ldots (app(x_K,x_1),\ldots,x_n)\ldots) \neq \unr} \\
%%  %%                  & \formula{\quad\quad \land\; app(\ldots (app(x_K,x_1),\ldots,x_n)\ldots) \neq \bad} \\
%%  %%                  & \text{ for every } (K{:}\forall\as @.@ \oln{\tau}{m} -> T\;\as) \in \Sigma \text{ and } m > n \\
%%  \textsc{AxPartB} & \formula{\forall \oln{x}{n} @.@ app(f_{ptr},\xs) \neq \unr} \\
%%                   & \formula{\quad\land\; app(f_{ptr},\xs) \neq \bad} \\
%%                   & \formula{\quad\land\; \forall \oln{y}{k} @.@ app(f_{ptr},\xs) \neq K(\ol{y})} \\
%%                   & \text{ for every } (f |-> \Lambda\as @.@ \lambda\oln{x{:}\tau}{m} @.@ u) \in P  \\
%%                   & \text{ and every } (K{:}\forall\as @.@ \oln{\tau}{k} -> T\;\as) \in \Sigma \text{ and } m > n  \\ \\ 
%%  \textsc{AxInj}   & \formula{\forall \oln{y}{n} @.@ \sel{K}{i}(K(\ys)) = y_i} \\ 
%%                   & \text{for every } (K{:}\forall\as @.@ \oln{\tau}{n} -> T\;\as) \in \Sigma \text{ and } i \in 1..n \\ \\
%% \end{array} \\
%% \ruleform{\Th{\Sigma}{P}^{\lcfZ}} \\ \\ 
%% \begin{array}{lll} 
%%  \textsc{AxCfA}   & \formula{\lcf{\unr} /\ \lncf{\bad}} \\
%%  \textsc{AxCfB}   & \formula{\forall \oln{x}{n} @.@ \lcf{K(\ol{x})} <=> \bigwedge\lcf{\ol{x}}} \\
%%                   & \text{ for every } (K{:}\forall\as @.@ \oln{\tau}{n} -> T\;\as) \in \Sigma
\end{array}\]


In the model, $\definable{\cdot}$ should be possible to define, as a
predicate on denotations. The disadvantage to this approach is that
arrow contracts will only hold for whatever is in your context, not
arbitrary expressions, which might be what we want, but might not be
modular enough.

The other {\em potential} problem (i.e. I have not yet checked) might be in the 
proof of admissibility of induction. 

And yet another potential problem is that as we incrementally extend our signature 
with new function definitions (and possibly contracts) previously defined contracts
may no longer hold. This is pretty bad for modularity.

\paragraph{Yet another possible solution}

A solution that seems somewhat more modular is based on the observation that, 
during the evaluation of a program there exists a {\em set} of terms (maybe infinite) that can
appear as arguments to other terms or functions. Our idea is to guard the arrow contracts so that
we do not quantify over any possible term (or denotation, in the translation) but rather only 
those that may appear as {\em arguments} in some application. We translate arrow contract as 
follows:
\[\begin{array}{l}
  \ctrans{\Sigma}{\Gamma}{e \in (x{:}\Ct_1) -> \Ct_2} = \\ 
  \qquad\qquad\quad 
\formula{\forall x @.@ arg(x) \land \ctrans{\Sigma}{\Gamma,x}{x \in \Ct_1} => \ctrans{\Sigma}{\Gamma,x}{e\;x \in \Ct_2}}
\end{array}\] 
where $arg(x)$ ensures that $x$ is the denotation of a term that will be passed as an argument to $e$. We'd need to define
a similar predicate on the evaluation relation, call it $Arg(e)$ and modify the program translation to thread the $arg(\cdot)$
predicate through. 



\section{Essential extensions}\label{sect:extensions}
  \subsection{Induction}\label{sect:induction}

\subsection{Minimization of countermodels}\label{sect:minimization}

For a query of the form ${\cal T}\;\land\;\neg(\ctrans{\Sigma}{P}{e \in \Ct})$, a satisfiability checker will search for
a model. When such a model exists, it will include tables for the function symbols in the formula. Notice that functions 
in FOL are total over the domain of the terms in the model. This means that function tables may be {\em infinite} if the 
terms in the model are infinite. Several (very useful!) axioms such as the discrimination axioms \textsc{AxDisjB} may in 
fact force the domains of functions operating e.g. on lists to be infinite. For instance consider the following devinitions:
\begin{code}
length [] = Z
length (x:xs) = S (length xs)

isZero Z = True
isZero _ = False
\end{code}
Suppose that we would like to check that 
   \[ @length@ \in \CF -> \{ x \mid @isZero@\;x\} \]
which is a falsifiable contract.  A satisfiability-based checker (such as \textsc{Eprover})
will simply diverge trying to construct a counter model for the negation of the above query.
Indeed the table for @length@ is infinite since @[]@ is always disjoint from @Cons x x@ for 
any @x@ and @xs@.

From a practical point of view this is not acceptable: After all, there exists a very simple 
counterexample that demonstrates the problem, e.g. @[Z]@, and we only need the 
functions of our program to be defined on a {\em finite} number of values (those that appear 
during the evaluation of this problematic counterexample) to be able to demonstrate 
the problem. We simply {\em do not care} about values that a function may take outside the set 
of expressions that appear during the finite evaluation of a counterexample.

To achieve this effect, we update our Prelude theory axioms as follows:
{\small
\[\setlength{\arraycolsep}{1pt}
\begin{array}{c}
%% \ruleform{\Th{\Sigma}{P}} \\ \\ 
\begin{array}{lll}
 \textsc{AxDisjA} & \formula{\bad \neq \unr}  \\ 
 \textsc{AxDisjB} & \formula{\forall \oln{x}{n}\oln{y}{m} @.@} \\ 
                  & \formula{\;\;\highlight{min(K(\ol{x}))\;\lor\;min(J(\ol{y}))} =>
                                  K(\ol{x}){\neq}J(\ol{y})} \\
                  & \text{ for every } (K{:}\forall\as @.@ \oln{\tau}{n} -> T\;\as) \in \Sigma \\ 
                  & \text{ and } (J{:}\forall\as @.@ \oln{\tau}{m} -> S\;\as) \in \Sigma \\
 %% \textsc{AxDisjCUnr} & \formula{\forall \oln{x}{n} @.@ \highlight{\neg min(\unr)}} \\ 
 %%                  & \text{ for every } (K{:}\forall\as @.@ \oln{\tau}{n} -> T\;\as) \in \Sigma \\ \\
 \textsc{AxDisjCBad} & \formula{\forall \oln{x}{n} @.@ K(\ol{x}) \neq \bad} \\ 
                  & \text{ for every } (K{:}\forall\as @.@ \oln{\tau}{n} -> T\;\as) \in \Sigma \\ \\

 \textsc{AxAppA}  & \formula{\forall \oln{x}{n} @.@ f(\ol{x}) = app(f_{ptr},\xs)} \\
                  & \text{ for every } (f |-> \Lambda\as @.@ \lambda\oln{x{:}\tau}{n} @.@ u) \in P \\
 %% \textsc{AxAppB}  & \formula{\forall \oln{x}{n} @.@ K(\ol{x}) = app(\ldots (app(x_K,x_1),\ldots,x_n)\ldots)} \\
 %%                  & \text{ for every } (K{:}\forall\as @.@ \oln{\tau}{n} -> T\;\as) \in \Sigma \\
 \textsc{AxAppC}  & \formula{\forall x, app(\bad,x) = \bad \; /\ \; app(\unr,x) = \unr}    \\ 
 \textsc{AxAppMin}& \formula{\highlight{\forall x, min(app(x,y)) => min(x)}} \\ 

 %% Not needed: we can always extend partial constructor applications to fully saturated and use AxAppC and AxDisjC
 %% \textsc{AxPartA} & \formula{\forall \oln{x}{n} @.@ app(\ldots (app(x_K,x_1),\ldots,x_n)\ldots) \neq \unr} \\
 %%                  & \formula{\quad\quad \land\; app(\ldots (app(x_K,x_1),\ldots,x_n)\ldots) \neq \bad} \\
 %%                  & \text{ for every } (K{:}\forall\as @.@ \oln{\tau}{m} -> T\;\as) \in \Sigma \text{ and } m > n \\
 %% \textsc{AxPartB} & \formula{\forall \oln{x}{n} @.@ app(f_{ptr},\xs) \neq \unr} \\
 %%                  & \formula{\quad\land\; app(f_{ptr},\xs) \neq \bad} \\
 %%                  & \formula{\quad\land\; \forall \oln{y}{k} @.@ app(f_{ptr},\xs) \neq K(\ol{y})} \\
 %%                  & \text{ for every } (f |-> \Lambda\as @.@ \lambda\oln{x{:}\tau}{m} @.@ u) \in P  \\
 %%                  & \text{ and every } (K{:}\forall\as @.@ \oln{\tau}{k} -> T\;\as) \in \Sigma \text{ and } m > n  \\ \\ 
 \textsc{AxInj}   & \formula{\forall \oln{y}{n} @.@ \highlight{min(K(\ys))\;\land\; min(y_i)}} \\ 
                  & \formula{\quad\qquad\qquad => \sel{K}{i}(K(\ys)) = y_i} \\ 
                  & \text{for every } (K{:}\forall\as @.@ \oln{\tau}{n} -> T\;\as) \in \Sigma \text{ and } i \in 1..n \\ \\

 \textsc{AxCfA}    & \formula{\lcf{\unr} /\ \lncf{\bad}} \\
 \textsc{AxCfMin}  & \formula{\highlight{\forall x @.@ \lcf{x} => min(x) \lor x = unr}} \\
 %% \textsc{AxCfB1}   & \formula{\forall \oln{x}{n} @.@ \bigwedge_i (\lcf{x_i}\lor \neg(min(x_i))} => \lcf{K(\ol{x})} \lor \neg(min(K(\ol{x}))) \\
 %%                   & \text{ for every } (K{:}\forall\as @.@ \oln{\tau}{n} -> T\;\as) \in \Sigma \\ 
 \textsc{AxCfB2}   & \formula{\forall \oln{x}{n} @.@ \lcf{K(\ol{x})}  => \bigwedge\lcf{\ol{x}}} \\
                   & \text{ for every } (K{:}\forall\as @.@ \oln{\tau}{n} -> T\;\as) \in \Sigma \\
 \textsc{AxNotMin} & \formula{\highlight{min(K(\oln{x}{n}))\land\neg\lcf{K(\oln{x}{n})}}} \\ 
                   & \formula{\quad\qquad\qquad \highlight{ => \bigvee_i (min(x_i)\land\neg\lcf{x_i})}}
\end{array}
\end{array}\]}


\begin{figure*}\small
\[\begin{array}{c} 
\ruleform{\ctrans{\Sigma}{\Gamma}{e \in \Ct} = \formula{\phi}} \\ \\ 
\prooftree
  \begin{array}{c}
   \etrans{\Sigma}{\Gamma}{e} = \formula{t} \quad
   \etrans{\Sigma}{\Gamma,x}{e'} = \formula{t'}
  \end{array}
  ------------------------------------------{CTransBase}
  \begin{array}{l}
   \ctrans{\Sigma}{\Gamma}{e \in \{(x{:}\tau) \mid e' \}} = 
  %% \Sigma;\Gamma |- e \in \{(x{:}\tau \mid e' \}
   \formula{\highlight{min(t) => (min(t'[t/x])}
                \land ((t = \unr) \lor (t'[t/x] = \unr) \lor (t'[t/x] = \True)))}
  \end{array}
  ~~~~~ 
  \begin{array}{c}
  \ctrans{\Sigma}{\Gamma,x}{x \notin \Ct_1} {=} \formula{\phi_1} \quad
  \ctrans{\Sigma}{\Gamma,x}{e\;x \in \Ct_2} {=} \formula{\phi_2}
  \end{array} 
  ------------------------------------------{CTransArr}
  \begin{array}{l} 
  \ctrans{\Sigma}{\Gamma}{e \in (x{:}\Ct_1) -> \Ct_2} = 
  \formula{\forall x @.@ \phi_1 \lor \phi_2}
  \end{array}
  ~~~~
  \begin{array}{c}
  \ctrans{\Sigma}{\Gamma}{e \in \Ct_1} = \formula{ \phi_1} \quad
  \ctrans{\Sigma}{\Gamma}{e \in \Ct_2} = \formula{ \phi_2}
  \end{array}
  ------------------------------------------{CTransConj}
  \ctrans{\Sigma}{\Gamma}{e \in \Ct_1 \& \Ct_2} = \formula{ \phi_1 /\ \phi_2}
  ~~~~~
  \etrans{\Sigma}{\Gamma}{e} =  \formula{t}
  -------------------------------------------{CTransCf}
  \ctrans{\Sigma}{\Gamma}{e \in \CF} = \formula{\highlight{min(t) => \lcf{t}}}
 \endprooftree  \\ \\ 
\ruleform{\ctrans{\Sigma}{\Gamma}{e \in \Ct} = \formula{\phi}} \\ \\ 
\prooftree
  \begin{array}{c}
   \etrans{\Sigma}{\Gamma}{e} = \formula{t} \quad
   \etrans{\Sigma}{\Gamma,x}{e'} = \formula{t'}
  \end{array}
  ------------------------------------------{NCTransBase}
  \begin{array}{l}
   \ctrans{\Sigma}{\Gamma}{e \notin \{(x{:}\tau) \mid e' \}} = 
  %% \Sigma;\Gamma |- e \in \{(x{:}\tau \mid e' \}
   \formula{\highlight{min(t) \land min(t'[t/x])}
             \land\;((t \neq \unr) \land ((t'[t/x] \neq \unr) \land (t'[t/x] \neq \True)))}
  \end{array}
  ~~~~~ 
  \begin{array}{c}
  \ctrans{\Sigma}{\Gamma,x}{x \in \Ct_1} {=} \formula{\phi_1} \quad
  \ctrans{\Sigma}{\Gamma,x}{e\;x \notin \Ct_2} {=} \formula{\phi_2}
  \end{array} 
  ------------------------------------------{NCTransArr}
  \begin{array}{l} 
  \ctrans{\Sigma}{\Gamma}{e \notin (x{:}\Ct_1) -> \Ct_2} = 
  \formula{\exists x @.@ \phi_1 \land \phi_2}
  \end{array}
  ~~~~
  \begin{array}{c}
  \ctrans{\Sigma}{\Gamma}{e \notin \Ct_1} = \formula{ \phi_1} \quad
  \ctrans{\Sigma}{\Gamma}{e \notin \Ct_2} = \formula{ \phi_2}
  \end{array}
  ------------------------------------------{NCTransConj}
  \ctrans{\Sigma}{\Gamma}{e \in \Ct_1 \& \Ct_2} = \formula{ \phi_1 \lor \phi_2}
  ~~~~~
  \etrans{\Sigma}{\Gamma}{e} =  \formula{t}
  -------------------------------------------{NCTransCf}
  \ctrans{\Sigma}{\Gamma}{e \notin \CF} = \formula{\highlight{min(t) \land \neg\lcf{t}}}
 \endprooftree 
\end{array}\]
\caption{Min-based contract elaboration}\label{fig:min-typing}
\end{figure*}

\subsubsection{min() as not unreachable}

TODO




\section{Implementation and practical experience}\label{sect:implementation}
  \dr{This is obviously chaotic, but I think we might want to keep some parts}
The implementation of the contracts checker has two components:

\begin{itemize}
    \item \textbf{Halo}, the Haskell to Logic translator, which takes
       GHC Core as input and translates function and data type
       definitions to first order logic as described in previous sections.

     \item \textbf{hcc}, the Haskell Contracts Checker, the user front
        end. Before giving halo the program to translate to FOL, hcc
        takes contracts, inputed in the Haskell source file as function
        definitions, identified by their type, and represents those in
        its own data type. Then we generate a theory accompanied with a
        conjecture for each contract, so the users can try them with
        their installed first order theorem provers.
\end{itemize}

Our tool is in under major development, but it already works off
actual Haskell programs.

\subsection{Why GHC Core}

The core language introduced in previous section \dr{which?} is very
similar to GHC Core so it is natural place to start. But what
differences do we have? Our language
\dr{does it have a name? $\mathcal{H}$?} is quite simplistic and does not allow local lambdas
or lets, and not case-expressions except at the top of a function
definition, so we lift all these definitions to the top, and introduce
a new function for each local lambda, let and case.  Since we are
doing an untyped translation, we remove all type information that is
available in this core language.
Rather than starting from scratch, using the GHC API gives us
desugaring, type inference and even optimisations!

We also do some optimisations for theorem provers, i.e. we try to
clausify as much as possible (theorem provers like CNF), and instead
of writing $x = K(xs) \rightarrow f(x) = e$, we simply write $f(K(xs)) = e$.

\subsection{Using the tool}

How does the user input contracts? We use HOAS and let the user define
contracts by primitive connectives, which can be extended by the user.
The connectives are defined as a GADT:

\begin{code}
data Contract t where
  (:->) :: Contract a
        -> (a -> Contract b)
        -> Contract (a -> b)
  Pred  :: (a -> Bool) -> Contract a
  CF    :: Contract a
  (:&:) :: Contract a -> Contract a -> Contract a
\end{code}

The connectives are @:->@ for dependent contract function space, @CF@
for crash-freedom (no catchable bottoms), @Pred@ for predication, and
@:&:@ for conjunction. A notable difference from the notation in
\cite{xu+:contracts}, the user of our library explicitly writes @CF@.
A useful derived connective is non-dependent function space:

\begin{code}
(-->) :: Contract a -> Contract b -> Contract (a -> b)
c1 --> c2 = c1 :-> \_ -> c2
\end{code}

%As one would expect, @:->@ and @-->@ are right-associve.  We can
%create contract combinators that are always satisfied, and never
%satisfied:
%
%\begin{code}
%any :: Contract a
%any = Pred (\ _ -> True)
%
%never :: Contract a
%never = Pred (error "never!")
%\end{code}

A contract is always associated to a function, so we wrap a contract
together with its function to a @Statement@:

\begin{code}
data Statement where
    (:::) :: a -> Contract a -> Statement
\end{code}

This allows us to wrap a function with a contract.  We also see how
GADTs can ensure that our contracts are well-typed.  One intended use
for our tool is to, if you have a function @f@ and it is partial, you
can state under which precoditions it is not partial, and then our
tool proves that it will not crash if you only give it arguments
satisfying these conditions. An example in this fashion about the
@head@ function is from the \cite{xu+:contracts}:

\begin{code}
head (x:xs) = x
head []     = error "empty list"

null [] = True
null xs = False

not True = False
not False = True

f . g = \x -> f (g x)

head_contract =
    head ::: CF :&: Pred (not . null) --> CF
\end{code}

Upon saving this file to, say, @Head.hs@ and running @hcc Head@, a
TPTP file named @Head.head_contract.tptp@ is generated. This can then
be run with your favourite automated theorem prover, and the above
contracts is easily verified by for instance @eprover@.

However, if we remove the precondition of crash-freedom to @head@,
like this:

\begin{code}
head_contract_broken =
    head ::: Pred (not . null) --> CF
\end{code}

a theorem prover capable of giving finite counter-models, like
@paradox@, will give you a counter model, which, upon examination,
will tell you that its input argument is indeed not an empty list, but
it is @:@ of some bad arguments. Note that we do not get a counter
example if we give @hcc@ the debug flag @--no-min@ that turns off the
min predicate introduced earlier \dr{where?}. The reason is that if we
do not use min, we only have infinite counter models because we need
to have an injective, but non-surjective function for @(:)@.

\subsection{Recursion}

We prove properties using fixed point induction. For recursive
functions, the tool then gives three files, one without induction, one
for the base case and step case. The typical situation is that the one
without induction does not pass because it lacks the induction
hypothesis, the base case always succeeds because that is what we
established our theory on: we need this to be admissible. Indeed, this
one does not need to be checked but it can serve as a good sanity
check of the tool. The step case can pass or fail, depending on if the
contract really holds, and if the induction hypothesis is strong
enough, and if we assume the right contracts.

\subsection{Higher order functions}

\begin{code}
all :: (a -> Bool) -> [a] -> Bool
all p []     = True
all p (x:xs) = p x && all p xs

filter :: (a -> Bool) -> [a] -> [a]
filter p [] = []
filter p (x:xs) | p x       = x : filter p xs
                | otherwise = filter p xs

filter_all_contr =
    filter ::: (CF --> CF) :-> \p ->
               CF :-> \xs ->
               (CF & Pred (all p))
\end{code}

This says that if filter takes a @p@ that is @CF --> CF@, and a
crash-free @xs@, then @filter p xs@ is @CF@ and satisfies @all p@,
i.e, the following returns @True@ or diverges:
@all p (filter p xs)@.
This contract is easily proved using fixed point induction.

% Note, however, that without induction we get a counter-example.

\subsection{Higher-higher order functions}

Our tool also deals with higher order functions containing
higher-order functions. Consider this function @withMany@ from the
@GHC@ component @Foreign.Util.Marshal@:
%\footnote{\url{http://hackage.haskell.org/packages/archive/base/latest/doc/html/Foreign-Marshal-Utils.html#v:withMany}}:

\begin{code}
-- Replicates a @withXXX@ combinator over a list of
-- objects, yielding a list of marshalled objects
--
withMany :: (a -> (b -> res) -> res)
         -- ^ withXXX combinator for one object
         -> [a]
         -- ^ storable objects
         -> ([b] -> res)
         -- ^ action on list of marshalled obj.s
         -> res
withMany _       []     f = f []
withMany withFoo (x:xs) f = withFoo x (\x2 ->
      withMany withFoo xs (\xs2 -> f (x2:xs2)))

\end{code}

For @withMany@, our tool proves
@(CF --> (CF --> CF) --> CF)@ @--> CF --> (CF --> CF) --> CF@.

\subsection{A small case-study about invariants}

We consider a somewhat non-standard way of expressing propositional
logic formulae:

\begin{code}
data Formula = And [Formula]
             | Or  [Formula]
             | Neg (Formula)
             | Implies (Formula) (Formula)
             | Lit Bool
\end{code}

One invariant that we are particularily interested in is that we
should never have two consecutive negations, and that the lists of
@And@ and @Or@ are of length $\ge$ 2. We can express that as an ordinary
Haskell predicate:

\begin{code}
invariant :: Formula -> Bool
invariant f = case f of
  And xs      -> properList xs && all invariant xs
  Or xs       -> properList xs && all invariant xs
  Neg Neg{}   -> False
  Neg x       -> invariant x
  Implies x y -> invariant x && invariant y
  Lit x       -> True

properList :: [a] -> Bool
properList []  = False
properList [_] = False
properList _   = True
\end{code}

Now, we have a recursive function that negates formula:

\begin{code}
neg :: Formula -> Formula
neg (Neg f)         = f
neg (And fs)        = Or (map neg fs)
neg (Or fs)         = And (map neg fs)
neg (Implies f1 f2) = neg f2 `Implies` neg f1
neg (Lit b)         = Lit b
\end{code}

We make a combinator saying what it means to retain a predicate:

\begin{code}
retain :: (a -> Bool) -> Contract (a -> a)
retain p = Pred p :-> \x -> Pred (\r -> p x && p r)
\end{code}

TODO: explain this. Now, since @neg@ uses @map@, we need to show that
@map@ can retain the invariant. We use @all@, introduced above, for
this:

\begin{code}
map_invariant = map ::: retain invariant
                    --> retain (all invariant)
\end{code}

Explicitly spelling out the definition of @retain@ in the statement
above would be tedious and error-prone, so we see the benefit of being
able to express contracts as a DSL.

Together with the statement @invariant ::: CF --> CF@, we
get that @neg@ retains the invariant thusly:

\begin{code}
neg_contr = neg ::: retain invariant
    `Using` map_invariant
    `Using` invariant_cf
\end{code}

We use @Using :: Statement -> Statement -> Statement@, another
constructor for @Statement@, which allows us to assume that other
contracts holds, when proving a complicated statement. For now, it's
the user's responsibility to prove these assumed contracts (for
instance, with our tool!), but one can imagine a more sophisticated
front-end which does this automatically.  Note that removing either
assumption yields theories that are satisfiable by 4-5 elements.

\subsection{Trimming}

We trim the theories as much as possible to only include exactly what
is needed to prove a property. Unnecessary function pointers, data
types and definitions for the current goal are not generated.

% \subsection{Example: shrink}
%
% Recall that @fromJust@ is the partial function @Maybe a -> a@, and consider
% this code:
%
% \begin{code}
% shrink :: (a -> a -> a) -> [Maybe a] -> a
% shrink op []     = error "Empty list!"
% shrink op [x]    = fromJust x
% shrink op (x:xs) = fromJust x `op` shrink op xs
% \end{code}
%
% Is this contract satisfied for it?
% \begin{code}
%     (CF --> CF --> CF) -->
%     (CF :&: Pred nonEmpty :&: Pred (all isJust)) --> CF
% \end{code}



\section{Discussion}\label{sect:discussion}
  \paragraph{Contracts that do not hold}
\label{ssect:countersat}

In practice, programmers will often propose contracts that do not hold. 
For example, consider the following definitions:
\begin{code}
  length []     = Z
  length (x:xs) = S (length xs)

  isZero Z = True
  isZero _ = False
\end{code}
Suppose that we would like to check the (false) contract:
   \[ @length@ \in \CF -> \{ x \mid @isZero@\;x\} \]
\emph{A satisfiability-based checker 
will simply diverge} trying to construct a counter model for the
negation of the above query; we have confirmed that this is indeed the
behaviour of several tools (Z3, Equinox, Eprover).  Why?  When a
counter-model exists, it will include tables for the function symbols
in the formula. Recall that functions in FOL are total over the domain
of the terms in the model. This means that function tables may be {\em
infinite} if the terms in the model are infinite. Several (very
useful!)  axioms such as the discrimination axioms \textsc{AxDisjC}
may in fact force the models to be infinite.

In our example, the table for @length@ is indeed infinite since @[]@ is
always disjoint from @Cons x xs@ for any @x@ and @xs@. Even if there
is a finitely-representable infinite model, the theorem prover may
search forever in the ``wrong corner'' of the model for a
counterexample.

From a practical point of view this is unfortunate; it is not
acceptable for the checker to loop when the programmer writes an
erroneous contract.  Tantalisingly, there exists a very simple
counterexample, e.g. @[Z]@, and that single small example is all the
programmer needs to see the falsity of the contract.

Addressing this problem is a challenging (but essential) 
direction for future work, and we are currently 
working on a modification of our theory that admits the denotational model, but 
also permits {\em finite models} generated from counterexample traces.
%% These ideas are reminiscent to the techniques that the Nitpick \dv{IS this right?} tool
%% uses for generating finite counterexamples in Isabelle. \dv{Someone please check!}.
If the theory can guarantee the existence of a finite model in case of a counterexample,
a finite model checker such as Paradox~\cite{paradox} will be able find it.

%% It is obviouly unacceptable for the system to go into a loop if
%% the programmer writes a bogus contract, and we have promising
%% preliminary results based on so-called ``minimisation'', and
%% finite counter-model generators such as Paradox \cite{koen}, but we
%% leave this for (absolutely essential) future work.

\paragraph{A tighter correspondence to operational semantics?}

Earlier work gave a declarative specfication of contracts using
\emph{operational semantics} \cite{xu+:contracts}.  In this paper we have
instead used a \emph{denotational semantics} for contracts (Figure~\ref{f:den-sem-contracts}).
It is natural to ask whether or not the two semantics are identical.

From computational adequacy, Theorem~\ref{thm:adequacy} we can easily state
the following theorem: 
\begin{corollary} Assume that $e$ and $\Ct$ contain no term variables and 
assume that $\ctrans{}{\cdot}{e \in \{x \mid e_p\}} = \formula{\phi}$. It is the case 
that $\langle D_\infty,{\cal I}\rangle \models \phi$ if and only iff either
$P \not|- e \Downarrow$ or $P \not|- e_p[e/x] \Downarrow$ or $P |- e_p[e/x] \Downarrow \True$. \end{corollary}
Hence, the operational and denotational semantics of \emph{predicate contracts} coincide.
However, the correspondence is not precise for \emph{dependent function contracts}.
Recall the operational definition of contract satisfaction for 
a function contract:
\[\begin{array}{l} 
   e \in (x{:}\Ct_1) -> \Ct_2 \text{ iff} \\
   \text{for all } e' \text{ such that } (e' \in \Ct_1) \text{ it is } e\;e' \in \Ct_2[e'/x]
\end{array}\] 
The denotational specification (Figure~\ref{f:den-sem-contracts}) 
says that for all denotations $d'$ such that
$d' \in \dbrace{\Ct_1}$, it is the case that 
$\dapp(\dbrace{e},d') \in \dbrace{\Ct_2}_{x |->d'}$. 
 
Alas there are {\em more} denotations than images of terms in $D_{\infty}$,
and that breaks the correspondence. Consider the program:
\[\begin{array}{lcl}
f_\omega = f_\omega \\
f \; (h{:}Bool->Bool->Bool) \\ 
\begin{array}{lll}
  & =   & @if@\;(h\;True\;True)\;\;\&\& \;\;not\;(h\;False\;False)\;@then@ \\
  &     & \quad @if@\;(h\;True\;f_\omega)\;\&\&\;(h\;f_\omega\;True)\;@then@\;@BAD@ \\
  &     & \quad @else@\;True \\
  &     & @else@\;True
\end{array}
\end{array}\]
Also consider now this candidate contract for $f$:
\[ @f@ \in \CF -> (\CF -> \CF -> \CF) -> \CF \]
Under the \emph{operational} definition of contract satisfaction, 
@f@ indeed satisfies the contract.
To reach @BAD@ we have to pass both conditionals.
The first ensures that $h$ evaluates at least one of its
arguments, while the second will diverge if either argument is evaluated.
Hence @BAD@ cannot be reached, and the contract is satisfied. 

However, \emph{denotationally} it is possible to have the classic 
parallel-or function, $por$, defined as 
follows\footnote{For convenience, 
we are using pattern matching notation instead of our 
language of domain theory combinators}:
\[\begin{array}{lcl}
  por\;\bot\;\bot & = & \bot \\ 
  por\;\bot\;True & = & True \\
  por\;True\;\bot & = & True \\ 
  por\;False\;False & = & False
\end{array}\] 
The rest of the equations (for @BAD@ arguments) are induced by
monotonicity and we may pick whatever boolean value we like when both
arguments are @BAD@.

Now, this is denotationally a $\CF -> \CF -> \CF$ function, but it
will pass \emph{both} conditionals, 
yielding @BAD@. Hence $app(@f@,por) = \injBad$, and @f@'s contract does not hold.
So we have a concrete case where an expression may satisfy its
contract operationally but not denotationally, because of the usual
loss of full abstraction: there are more tests than programs in the denotational world. 
Due to contra-variance
we expect that the other inclusion will fail too. 

This is not a serious problem in practice.
After all the two definitions mostly coincide, and
they precisely coincide in the base case.  At the end of the day, we
are interested in whether $@main@ \in \CF$, and we have proven that if
is crash-free denotationally, it is definitely crash-free in any
operationally-reasonable term.

Finally, is it possible to define an operational model for our FOL theory that interpreted
equality as contextual equivalence? Probably this could be made to work, although we believe
that the formal clutter from syntactic manipulation of terms could be worse than the current
denotational approach. 


\paragraph{Polymorphic crash-freedom}

Observe that our axiomatisation of crash-freedom in Figure~\ref{fig:prelude} 
includes only axioms for data constructors. In fact, our denotational interpretation
$\Fcf^{\infty}$ allows more axioms, such as:
\[\begin{array}{l}
    \forall x y @.@ \lcf{x} \land \lcf{y} => \lcf{app(x,y)}
\end{array}\] 
This axiom is useful if we wish to give directly a $\CF$ contract to a value of 
arrow type. For instance, instead of specifying that @map@ satisfies the contract
$(\CF -> \CF) -> \CF -> \CF$ one may want to say that it satisfies the contract
$\CF -> \CF -> \CF$. With the latter contract we need the previous axiom to be 
able to apply the function argument of @map@ to a crash-free value and get a 
crash-free result. 

In some situations, the following axiom might be beneficial as well:
\[\begin{array}{l}
    (\forall \xs @.@ \lcf{f(\xs)}) => \lcf{f_{ptr}}
\end{array}\]
If the result of applying a function to any possible argument is crash-free then 
so is the function pointer. This allows us to go in the inverse direction as before, 
and pass a function pointer to a function that expects a $\CF$ argument. However notice
that this last axiom introduces a quantified assumption, which might lead to significant
efficiency problem.

Ideally we would like to say that $\dbrace{\CF} = \dbrace{\CF \rightarrow \CF}$,
but that is not quite true.  In particular, 
\[\begin{array}{l}
   (\forall x @.@ \lcf{app(y,x)}) => \lcf{y}
\end{array}\]
is {\em not} valid in the denotational model. For instance consider the
value $\injK{K}{\injBad}$ for $y$. The left-hand side is going to always 
be true, because the application is ill-typed and will yield $\bot$, but $y$ 
is not itself crash-free.



\section{Future work}\label{sect:future}
  % Integers
% SMT 2.0
% Printing countermodels
% (Typeclasses)

Static verification for functional programming languages seems an
under-studied (compared to the imperative world) and very 
promising area of research.
In practical terms, our most immediate goal is to
ensure that we can find finite counter-examples quickly,
and present them comprehensibly to the user, rather allowing the
theorem prover to diverge.
As mentioned in Section~\ref{ssect:countersat} we have well-developed ideas
for how to do this.  It would also be interesting 
to see if \emph{triggers} in SMT 2.0 could also be used to support that goal. 

We would like to add support for primitive data types, such as
@Integer@, using theorem provers such as @T-SPASS@ to deal with the
@tff@ (typed first-order arithmetic) part of TPTP. Another approach might be to
generate theories in the SMT 2.0 format, understood by Z3, which
has support for integer arithmetic and more.  Another important
direction is finding ways to split our big verification goals into
smaller ones that can be proven significantly faster. Finally, we
would like to investigate whether we can automatically strengthen
contracts to be used as induction hypotheses in inductive proofs,
deriving information from failed attempts.

%A way of presenting countermodels given by @paradox@ in an easily
%understandable way for the user would be helpful.
% quote Reasoning with Triggers?


\bibliographystyle{plainnat}
\bibliography{hcc-popl}

\end{document}

%% \begin{abstract}
%% The Glasgow Haskell Compiler is an optimizing
%% compiler that expresses and manipulates first-class equality proofs in
%% its intermediate language.  We describe a simple, elegant technique that
%% exploits these equality proofs to support \emph{deferred type errors}.
%% The technique requires us to treat equality proofs as possibly-divergent
%% terms; we show how to do so without losing either soundness or
%% the zero-overhead cost model that the programmer expects.
%% \end{abstract}

%% \category{D.3.3}{Language Constructs and Features}{Abstract data types}
%% \category{F.3.3}{Studies of Program Constructs}{Type structure}

%% \terms{Design, Languages}

%% \keywords{Type equalities, Deferred type errors, System FC}

\section{Denotational semantics}


%% \begin{lemma}[Evaluation preserves equality]
%% If $\Sigma;\cdot |- e : \tau \rightsquigarrow t$ and 
%%    $\Sigma |- D \rightsquigarrow \phi_{\Sigma,D}$ and 
%%    $D |- e \Downarrow w$ then
%%    $\Sigma;\cdot |- w : \tau \rightsquigarrow s$ and $\Th{\Sigma}{D} /\ \phi_{\Sigma,D} |- t = s$.
%% \end{lemma}
%% \begin{proof} By induction on the evaluation $\Sigma |- e \Downarrow w$. \end{proof}


%% \begin{lemma}[Logic deduces sound value equalities]
%% Assume that $\Sigma;\cdot |- w : \tau \rightsquigarrow t$ and 
%% $D |- value(w)$ and $\Sigma |- D \rightsquigarrow \phi_{\Sigma,D}$. 
%% Then
%% \begin{enumerate*} 
%%   \item If $\Th{\Sigma}{D} /\ \phi_{\Sigma,D} |- t = \bad$ then $w = @BAD@$.
%%   \item If $\Th{\Sigma}{D} /\ \phi_{\Sigma,D} |- t = K(\ol{t})$ then $w = K[\taus](\ol{e})$, such 
%%         that $\Sigma;\cdot |- \ol{e : \tau} \rightsquigarrow \ol{s}$, and $\Th{\Sigma}{D} /\ \phi_{\Sigma,D} |- \ol{t = s}$.
%%   \item $\Th{\Sigma}{D} /\ \phi_{\Sigma,D} |- t \neq \unr$.
%% \end{enumerate*}
%% \end{lemma}
%% \begin{proof}
%% The proof of all three cases is by inversion on the $D |- value(w)$ derivation, 
%% apealling to the disjointness axioms.
%% %% \begin{enumerate*}
%% %%   \item By inversion on the $D |- value(w)$ derivation. In the case of \rulename{VBad} we are done.
%% %%   The case of \rulename{VFun} cannot happen, by the axiom set \rulename{AxPartB}. The case of \rulename{VCon} 
%% %%   cannot happen either: If the application is saturated then \rulename{AxDisjC} shows it is impossible; if it
%% %%   is not saturated we can always extend it and use \rulename{AxAppC} and \rulename{AxDisjC}. 
%% %%   \item Again by inversion on $D |- value(w)$ derivation. The case of \rulename{VBad} is easy. The case for 
%% %%   \rulename{VCon} follows by injectivity of constructors. The case of \rulename{VFun} can't happen by 
%% %%   \rulename{AxPartB}.
%% %%   \item Direct inversion on $D |- value(w)$, and using disjointness axioms.
%% %% \end{enumerate*} 
%% \end{proof}
 
%% Basic soundness will be stated as follows.
%% \begin{theorem}
%% If we have that
%% \begin{enumerate*} 
%%   \item $\Sigma;\cdot |- e : \tau$ and $\Sigma;\cdot |- \Ct : \tau$
%%   \item $\Sigma |- D \rightsquigarrow \phi_{\Sigma,D}$
%%   \item $\Sigma;\cdot |- e \in \Ct \rightsquigarrow \phi$
%% \end{enumerate*}
%% and $\Th{\Sigma}{D} /\ \phi_{\Sigma,D} /\ \neg \phi$ is unsatisfiable then $\Sigma;D |- e \in \Ct$.
%% \end{theorem}
%% \begin{proof}
%%  {\bf TODO}
%% \end{proof}

%% A remark: a formula $\phi$ is unsatisfiable iff $\neg \phi$ is valid in FOL. Hence, if 
%% $\Th{\Sigma}{D} /\ \phi_{\Sigma,D} /\ \neg \phi$ is unsatisfiable then 
%% $\neg (\Th{\Sigma}{D} /\ \phi_{\Sigma,D}) \lor \phi$ must be valid, and by completeness of FOL, 
%% $\Th{\Sigma}{D} /\ \phi_{\Sigma,D} |- \phi$.  

\section{Denotational semantics as FOL models} 



%% \begin{figure}\small
%% \[\begin{array}{c} 
%% %% \ruleform{ \dtrans{\Sigma}{d} = \formula{\phi} } \\ \\
%% %% \prooftree
%% %%   \begin{array}{c}
%% %%   (f{:}\forall\oln{a}{n} @.@ \oln{\tau}{m} -> \tau) \in \Sigma \quad 
%% %%   \etrans{\Sigma}{\ol{a},\ol{x{:}\tau}}{e} = \formula{t}
%% %%   \end{array}
%% %%   -------------------------------------------------------------------{TFDef}
%% %%   \dtrans{\Sigma}{(f |-> \Lambda\oln{a}{n} @.@ \lambda\oln{x{:}\tau}{m} @.@ e)} =  \formula{ (\forall x @.@ f(\oln{x}{m}) = t) }
%% %%   ~~~~~ 
%% %%   \begin{array}{l}
%% %%   (f{:}\forall\oln{a}{n} @.@ \oln{\tau}{m} -> \tau) \in \Sigma \quad 
%% %%   \etrans{\Sigma}{\ol{a},\ol{x{:}\tau}}{e} = \formula{t} \\
%% %%   constrs(\Sigma,T) = \ol{K} \\
%% %%   \text{for each branch}\;(K\;\oln{y}{l} -> e') \\
%% %%   \quad \begin{array}{l}
%% %%            (K{:}\forall \cs @.@ \oln{\sigma}{l} -> T\;\oln{c}{k}) \in \Sigma \\
%% %%            \etrans{\Sigma}{\ol{a},\ol{x{:}\tau},\ol{y{:}\sigma[\taus/\cs]}}{e'} = \formula{ t_K }
%% %%         \end{array}
%% %%   \end{array}
%% %%   -------------------------------------------------------------------{TCaseDef}
%% %%   \begin{array}{l}
%% %%    \dtrans{\Sigma}{(f |-> \Lambda\oln{a}{n} @.@ \lambda\oln{x{:}\tau}{m} @.@ @case@\;e\;@of@\;\ol{K\;\ol{y} -> e'})} = \\
%% %%    \quad \formula{ \begin{array}{lll} \forall \oln{x}{m} @.@ & \hspace{-7pt} (t = \bad /\ f(\ol{x}) = \bad)\; \lor \\ 
%% %%                                                                     & \hspace{-7pt}(f(\ol{x}) = \unr)\;\lor \\ 
%% %%                                                                     & \hspace{-7pt}(\bigvee(t = K(\oln{{\sel{K}{i}}(t)}{i\in 1..l})\;/\ \\
%% %%                                                                     & \hspace{-5pt}\quad f(\ol{x}) = t_K[\oln{\sel{K}{i}(t)}{i\in 1..l}/\ol{y}]))
%% %%                                                  \end{array}
%% %%                         }
%% %% \end{array}
%% %% \endprooftree  \\ \\ 
%% \end{array}\]
%% \caption{Definition elaboration to FOL}\label{fig:typing}
%% \end{figure}




{\bf DV: So basically this is Simon's strategy of side-stepping the lack of full abstraction
and the associated problems with it: In the end of the day we only care about base contracts,
in fact really only about the contract ``is this program crash-free'', so we don't have to make
a big fuss about higher-order contracts and their operational semantics. We have to motivate it
carefully and also be clear that for the intellectually curious reader who really wants to know what statement we have proved for a function contract when the prover says ``unsat'' we might want to give a full definition of the denotational meaning of contracts including both base and higher-order. I think we do not have the time luxury to look for more elaborate solutions (such as definable denotations and all that crazy stuff) to match the operational and the denotational semantics for higher-order contracts. Fullstop.}


\section{Minimizing countermodels}



\section{Min as unreachable}

 In fact we may take
one step further and equate all the non-interesting values of the domain to $\bot$.

To achieve this effect, we update our Prelude theory axioms as follows:
{\small
\[\setlength{\arraycolsep}{1pt}
\begin{array}{c}
%% \ruleform{\Th{\Sigma}{P}} \\ \\ 
\begin{array}{lll} 
 \textsc{AxDisjA} & \formula{\bad \neq \unr}  \\ 
 \textsc{AxDisjB} & \formula{\forall \oln{x}{n}\oln{y}{m} @.@} \\ 
                  & \formula{\;\;\highlight{K(\ol{x}){\neq}\unr\;\lor\;J(\ol{y}){\neq}\unr} =>
                                  K(\ol{x}){\neq}J(\ol{y})} \\
                  & \text{ for every } (K{:}\forall\as @.@ \oln{\tau}{n} -> T\;\as) \in \Sigma \\ 
                  & \text{ and } (J{:}\forall\as @.@ \oln{\tau}{m} -> S\;\as) \in \Sigma \\
 %% \textsc{AxDisjCUnr} & \formula{\forall \oln{x}{n} @.@ \highlight{\neg min(\unr)}} \\ 
 %%                  & \text{ for every } (K{:}\forall\as @.@ \oln{\tau}{n} -> T\;\as) \in \Sigma \\ \\
 \textsc{AxDisjCBad} & \formula{\forall \oln{x}{n} @.@ K(\ol{x}) \neq \bad} \\ 
                  & \text{ for every } (K{:}\forall\as @.@ \oln{\tau}{n} -> T\;\as) \in \Sigma \\ \\

 \textsc{AxAppA}  & \formula{\forall \oln{x}{n} @.@ f(\ol{x}) = app(f_{ptr},\xs)} \\
                  & \text{ for every } (f |-> \Lambda\as @.@ \lambda\oln{x{:}\tau}{n} @.@ u) \in P \\
 %% \textsc{AxAppB}  & \formula{\forall \oln{x}{n} @.@ K(\ol{x}) = app(\ldots (app(x_K,x_1),\ldots,x_n)\ldots)} \\
 %%                  & \text{ for every } (K{:}\forall\as @.@ \oln{\tau}{n} -> T\;\as) \in \Sigma \\
 \textsc{AxAppC}  & \formula{\forall x, app(\bad,x) = \bad \; /\ \; app(\unr,x) = \unr}    \\ \\
 %% Not needed: we can always extend partial constructor applications to fully saturated and use AxAppC and AxDisjC
 %% \textsc{AxPartA} & \formula{\forall \oln{x}{n} @.@ app(\ldots (app(x_K,x_1),\ldots,x_n)\ldots) \neq \unr} \\
 %%                  & \formula{\quad\quad \land\; app(\ldots (app(x_K,x_1),\ldots,x_n)\ldots) \neq \bad} \\
 %%                  & \text{ for every } (K{:}\forall\as @.@ \oln{\tau}{m} -> T\;\as) \in \Sigma \text{ and } m > n \\
 %% \textsc{AxPartB} & \formula{\forall \oln{x}{n} @.@ app(f_{ptr},\xs) \neq \unr} \\
 %%                  & \formula{\quad\land\; app(f_{ptr},\xs) \neq \bad} \\
 %%                  & \formula{\quad\land\; \forall \oln{y}{k} @.@ app(f_{ptr},\xs) \neq K(\ol{y})} \\
 %%                  & \text{ for every } (f |-> \Lambda\as @.@ \lambda\oln{x{:}\tau}{m} @.@ u) \in P  \\
 %%                  & \text{ and every } (K{:}\forall\as @.@ \oln{\tau}{k} -> T\;\as) \in \Sigma \text{ and } m > n  \\ \\ 
 \textsc{AxInj}   & \formula{\forall \oln{y}{n} @.@ \highlight{K(\ys) \neq \unr\;\land\; y_i \neq \unr}} \\ 
                  & \formula{\quad\qquad\qquad => \sel{K}{i}(K(\ys)) = y_i} \\ 
                  & \text{for every } (K{:}\forall\as @.@ \oln{\tau}{n} -> T\;\as) \in \Sigma \text{ and } i \in 1..n \\ \\
 \textsc{AxCfA}   & \formula{\lcf{\unr} /\ \lncf{\bad}} \\
 \textsc{AxCfB1}  & \formula{\forall \oln{x}{n} @.@ \bigwedge\lcf{\ol{x}}} => \lcf{K(\ol{x})} \\
                  & \text{ for every } (K{:}\forall\as @.@ \oln{\tau}{n} -> T\;\as) \in \Sigma \\ 
 \textsc{AxCfB2}  & \formula{\forall \oln{x}{n} @.@ \lcf{K(\ol{x})}\;\highlight{\land\;K(\ol{x}) \neq \unr} => \bigwedge\lcf{\ol{x}}} \\
                  & \text{ for every } (K{:}\forall\as @.@ \oln{\tau}{n} -> T\;\as) \in \Sigma
\end{array}
\end{array}\]}


\begin{figure}\small
\[\begin{array}{c}
\ruleform{\utrans{\Sigma}{\Gamma}{t \sim u} = \formula{\phi}} \\ \\ 
\prooftree
   \begin{array}{c} \ \\ \ \\ 
   \etrans{\Sigma}{\Gamma}{e} = \formula{t}
   \end{array}
   ----------------------------------------{TUTm}
   \begin{array}{l} 
   \utrans{\Sigma}{\Gamma}{s \sim e } = \formula{(s = t) \lor \highlight{\neg min(s)}} \ \\ \ \\ \ \\ 
   \end{array}
   ~~~~~
  \begin{array}{l}
  \etrans{\Sigma}{\Gamma}{e} = \formula{t} \quad
  constrs(\Sigma,T) = \ol{K} \\
  \text{for each branch}\;(K\;\oln{y}{l} -> e') \\
  \begin{array}{l}
           (K{:}\forall \cs @.@ \oln{\sigma}{l} -> T\;\oln{c}{k}) \in \Sigma \text{ and }
           \etrans{\Sigma}{\Gamma,\ol{y}}{e'} = \formula{ t_K }
  \end{array}
  \end{array}
  ------------------------------------------{TUCase}
  {\setlength{\arraycolsep}{1pt} 
  \begin{array}{l}
  \utrans{\Sigma}{\Gamma}{s \sim @case@\;e\;@of@\;\ol{K\;\ol{y} -> e'}} = \\
  \;\;\formula{ \begin{array}{l} 
     \highlight{min(s)} => \\
     \begin{array}{ll}
          ( & \highlight{min(t)}\;\land \\
            & (t = \bad => s = \bad)\;\land \\ 
            & (\forall \ol{y} @.@ t = K_1(\ol{y}) => s = t_{K_1})\;\land \ldots \land \\
            & (t \neq \bad\;\land\;t \neq K_1(\oln{{\sel{K_1}{i}}(t)}{})\;\land\;\ldots => s = \unr) \\
          )
%% (t = \bad /\ s = \bad)\;\lor\;(s = \unr)\;\lor \\
%%                                 \quad      \bigvee(t = K(\oln{{\sel{K}{i}}(t)}{}) \land
%%                                            s = t_K[\oln{\sel{K}{i}(t)}{}/\ol{y}])
                   \end{array}
     \end{array}}
  \end{array}}
  %% {       \setlength{\arraycolsep}{2pt} 
  %% \begin{array}{l}
  %% \utrans{\Sigma}{\Gamma}{s \sim @case@\;e\;@of@\;\ol{K\;\ol{y}{->}e'}} = \\
  %% \;\;\formula{
  %%      \begin{array}{l} (\highlight{s{=}\unr})\;\lor \\ 
  %%                           \;\; (\highlight{min(s) => min(t)}\;\land  \\
  %%                           \quad((t = \bad /\ s = \bad)\;\lor \\
  %%                           \quad\quad \bigvee(t = K(\oln{{\sel{K}{i}}(t)}{}) \land
  %%                                          s = t_K[\oln{\sel{K}{i}(t)}{}/\ol{y}])))
  %%                  \end{array}
  %%          }
  %% \end{array}}
\endprooftree
\end{array}\]
\caption{Minimality-enabled definition translation}\label{fig:min-def-trans-min}
\end{figure}



We will explain the modifications to the axiomatization in more detail in later sections.
%% In other words, we ensure that constructor applications are disjoint
%% only for values we are interested in. We will explain each axiom separately later. 
%% Intuitively we wish to equate all terms that we are not interested in to $\unr$. We 
%% can never be interested in $\unr$ in the intended model because that means that during
%% the evaluation of a term, which completed, we encountered a divergent term -- clearly a 
%% contradiction!
What about function definitions? Figure~\ref{fig:etrans} has to be modified slightly as well, 
as Figure~\ref{fig:min-def-trans} shows.

\begin{figure}\small
\[\begin{array}{c}
\ruleform{\utrans{\Sigma}{\Gamma}{t \sim u} = \formula{\phi}} \\ \\ 
\prooftree
   \begin{array}{c} \ \\ \ \\ 
   \etrans{\Sigma}{\Gamma}{e} = \formula{t}
   \end{array}
   ----------------------------------------{TUTm}
   \begin{array}{l} 
   \utrans{\Sigma}{\Gamma}{s \sim e } = \formula{(s = t) \lor \highlight{s = \unr}} \ \\ \ \\ \ \\ 
   \end{array}
   ~~~~~
  \begin{array}{l}
  \etrans{\Sigma}{\Gamma}{e} = \formula{t} \quad
  constrs(\Sigma,T) = \ol{K} \\
  \text{for each branch}\;(K\;\oln{y}{l} -> e') \\
  \begin{array}{l}
           (K{:}\forall \cs @.@ \oln{\sigma}{l} -> T\;\oln{c}{k}) \in \Sigma \text{ and }
           \etrans{\Sigma}{\Gamma,\ol{y}}{e'} = \formula{ t_K }
  \end{array}
  \end{array}
  ------------------------------------------{TUCase}
  {\setlength{\arraycolsep}{1pt} 
  \begin{array}{l}
  \utrans{\Sigma}{\Gamma}{s \sim @case@\;e\;@of@\;\ol{K\;\ol{y} -> e'}} = \\
  \;\;\formula{ \begin{array}{l} 
     \highlight{s = \unr}\;\lor \\
     \begin{array}{ll}
          ( & \highlight{(t \neq \unr)}\;\land \\
            & (t = \bad => s = \bad)\;\land \\ 
            & (\forall \ol{y} @.@ t = K_1(\ol{y}) => s = t_{K_1})\;\land \ldots \land \\
            & (t = \bad\;\lor\;t = K_1(\oln{{\sel{K_1}{i}}(t)}{})\;\lor\;\ldots) \\ 
          )
%% (t = \bad /\ s = \bad)\;\lor\;(s = \unr)\;\lor \\
%%                                 \quad      \bigvee(t = K(\oln{{\sel{K}{i}}(t)}{}) \land
%%                                            s = t_K[\oln{\sel{K}{i}(t)}{}/\ol{y}])
                   \end{array}
     \end{array}}
  \end{array}}
  %% {       \setlength{\arraycolsep}{2pt} 
  %% \begin{array}{l}
  %% \utrans{\Sigma}{\Gamma}{s \sim @case@\;e\;@of@\;\ol{K\;\ol{y}{->}e'}} = \\
  %% \;\;\formula{
  %%      \begin{array}{l} (\highlight{s{=}\unr})\;\lor \\ 
  %%                           \;\; (\highlight{min(s) => min(t)}\;\land  \\
  %%                           \quad((t = \bad /\ s = \bad)\;\lor \\
  %%                           \quad\quad \bigvee(t = K(\oln{{\sel{K}{i}}(t)}{}) \land
  %%                                          s = t_K[\oln{\sel{K}{i}(t)}{}/\ol{y}])))
  %%                  \end{array}
  %%          }
  %% \end{array}}
\endprooftree
\end{array}\]
\caption{Minimality-enabled definition translation}\label{fig:min-def-trans}
\end{figure}

%% \\ \\ 
%% \ruleform{ \Dtrans{\Sigma}{P} = \formula{\phi}} \\ \\ 
%% \prooftree
%%      \begin{array}{l}       
%%        \text{for each} (f |-> \Lambda\oln{a}{n} @.@ \lambda\oln{x{:}\tau}{m} @.@ u) \in P \\ 
%%           \quad \utrans{\Sigma}{\ol{x}}{f(\ol{x}) \sim u} = \formula{\phi}
%%      \end{array}
%%      --------------------{TDefs}
%%      \Dtrans{\Sigma}{P} = \bigwedge_{P} \formula{\forall \ol{x} @.@ \phi}
%% \endprooftree 

Now operationally we may instrument the evaluation relation to keep track of the set of 
closed terms that appear during evaluation. The instrumented relation appears in 
Figure~\ref{fig:opsem-instrumented}. Observe that if $P |- e \Downarrow w \curly S$ then 
$S$ is a {\em finite set} of terms.


\begin{figure}\small
\[\begin{array}{c} 
\ruleform{P |- e \Downarrow v \curly S} \\ \\ 
\prooftree
\begin{array}{c} \ \\ 
\end{array}
%% \begin{array}{c}
%% (f |-> \Lambda\ol{a} @.@ \lambda\oln{x{:}\tau}{m} @.@ u) \in P \\
%% P |- e_1 \Downarrow f\;[\taus]\;\oln{e}{m-1} \curly S_1 \\ 
%% P |- u[\ol{\tau}/\ol{a}][\ol{e},e_2/\ol{x}] \Downarrow w \curly S
%% \end{array}
%% ------------------------------------{EBeta}
%% P |- e_1\;e_2 \Downarrow w 
  S = heads(v)
-------------------------------------{EVal}
P |- v \Downarrow v \curly S
~~~~
\begin{array}{c}
(f |-> \Lambda\ol{a} @.@ \lambda\oln{x{:}\tau}{m} @.@ u) \in P \\
P |- u[\ol{\tau}/\ol{a}][\ol{e}/\ol{x}] \Downarrow v \curly S_1 \\ 
S_2 = heads(f[\ol{\tau}]\;\oln{e}{m}) 
\end{array}
-------------------------------------{EFun}
P |- f[\ol{\tau}]\;\oln{e}{m} \Downarrow v \curly S_1 \cup S_2
~~~~~
\begin{array}{c}  
P |- e_1 \Downarrow v_1 \curly S_1 \quad
P |- v_1\;e_2 \Downarrow w \curly S_2
\end{array}
------------------------------------------------{EApp}
P |- e_1\;e_2 \Downarrow w \curly S_1 \cup S_2 \cup \{ e_1\;e_2 \}
~~~~~
\begin{array}{c}  
P |- e_1 \Downarrow @BAD@ \curly S 
\end{array}
------------------------------------------------{EBadApp}
P |- e_1\;e_2 \Downarrow @BAD@ \curly S \cup \{ e_1\;e_2 \} 
\endprooftree \\ \\ 
\ruleform{heads(e) = S} \\ \\ 
\begin{array}{lcl}
   heads(f\;[\ol{\tau}]) & = & \{ f\;[\ol{\tau}] \} \\
   heads(e_1\;e_2)       & = & \{ e_1\;e_2 \} \cup heads(e_1) \\
   heads(\_)            & = & \emptyset 
\end{array} \\ \\
\ruleform{P |- u \Downarrow v \curly S} \\ \\
\prooftree
P |- e \Downarrow v \curly S 
-------------------------------------{EUTm}
P |- e \Downarrow v \curly S 
~~~~~
\begin{array}{c}
P |- e \Downarrow K_i[\ol{\sigma}_i](\ol{e}_i) \curly S_1 \quad
P |- e'_i[\ol{e}_i/\ol{y}_i] \Downarrow w \curly S_2 
\end{array}
------------------------------------{ECase}
P |- @case@\;e\;@of@\;\ol{K\;\ol{y} -> e'} \Downarrow w \curly S_1 \cup S_2
~~~~~
\begin{array}{c}
P |- e \Downarrow @BAD@ \curly S \\
\end{array}
------------------------------------{EBadCase}
P |- @case@\;e\;@of@\;\ol{K\;\ol{y} -> e'} \Downarrow @BAD@ \curly S
%% \begin{array}{c}
%% (f |-> \Lambda\ol{a} @.@ \lambda\oln{x{:}\tau}{m} @.@ @case@\;e\;@of@\;\ol{K\;\ol{y} -> e'}) \in D \\
%% D |- e[\ol{\tau}/\ol{a}][\ol{e}/\ol{x}] \Downarrow @BAD@ \\
%% \end{array}
%% -------------------------------------{EBadCase}
%% D |- f[\ol{\tau}]\;\oln{e}{m} \Downarrow @BAD@
\endprooftree
\end{array}\]
\caption{Redex-instrumented operational semantics}\label{fig:opsem-instrumented}
\end{figure}

\subsection{The intended min-imal model}

Our goal is then going to be to establish the following result, stated in non-technical terms:
\begin{quote}
If there exists a counterexample to a contract, then the negation of the contract-translation
formula is satisfiable not only on $\langle D_\infty,{\cal I}\rangle$ but it also has a {\em finite} 
model. That finite model is a model of our minimality-enabled theory. 
\end{quote}

We start unfolding the story. For a given program $P$ in a signature $\Sigma$ we have already 
shown how to construct $D_\infty$ and how to give interpretations ${\cal I}$ to a first-order 
vocabulary. Let us assume that the program and signature contains a polymorpic $undefined$ 
function, for convenience $undefined |-> udefined$. This is a realistic assumption to make 
(e.g. it comes in the standard Haskell prelude).

Assume now that we are given a formula $\phi$ defined as: 
\[  \phi = \ctrans{\Sigma}{\cdot}{e \in \Ct_1 -> \ldots \Ct_n -> @B@} \] 
for @B@ a base contract. Assume moreover that there exist $\oln{e}{n}$, closed for the
program $P$, such that for each $e_i$ it is true that:
\[\interp{\Ct_i}{\dbrace{P}^\infty}{\cdot}(\interp{e_i}{\dbrace{P}^\infty}{\cdot})\]. 
Assume however that it is {\em not} the case that
\[\interp{{\tt B}}{\dbrace{P}^\infty}{\cdot}(\interp{e\;\oln{e}{n}}{\dbrace{P}^\infty}{\cdot})\]
There are two cases for the base constract @B@:
\begin{itemize}
  \item Let us now consider the case where @B@ = $\{ x \mid e_p \}$. By adequacy it must
  be that: $P |- e\;\ol{e} \Downarrow w \curly S_1$ for some $w$ and set $S_1$ and moreover
  $P |- e_p[e\;ol{e}/x] \Downarrow \{ @BAD@, False \} \curly S_2$ for some set $S_2$. 

  Of course the following lemma is true:
  \begin{lemma}\label{lem:curly} 
    If $P |- e \Downarrow w \curly S$ then $S$ is a finite set. Moreover, 
    for every $e' \in S$ there exists $w$ such that $P |- e' \Downarrow w$.
  \end{lemma}
  Moreover we have:
  \begin{lemma}\label{lem:bot-not-redex} 
     If $P |- e \Downarrow w \curly S$ then 
     $\bot \notin \interp{S}{\dbrace{P}^{\infty}}{\cdot}$. 
  \end{lemma}
  \begin{proof} If $\bot \in \interp{S}{\dbrace{P}^{\infty}}{\cdot}$ then there exists
  a term $e \in S$ such that $\interp{e}{\dbrace{P}^{\infty}}{\cdot} = \bot$. This means
  that $P |- e \not\Downarrow$ but that is a contradiction to $e \in S$ by 
  Lemma~\ref{lem:curly}.
  \end{proof}

  Let us now define the {\em minimal sets} operationally and denotationally:

  \[\begin{array}{lcl}
           M        & \triangleq & S_1 \cup S_2 \\
           {\cal M} & \triangleq & \interp{S_1\cup S_2}{\dbrace{P}^{\infty}}{\cdot}
  \end{array}\]
  Consider now the function $\mu : D_\infty -> D_\infty$ defined as: 
  \[\begin{array}{lcl} 
        \mu(d) & \triangleq & \left\{ \begin{array}{ll} 
                   d           & \text{when } \unroll(d) = \ret(\inj{bad}(1)) \\
                   d           & \text{when } d \in \Min \\ 
                   \bot        & \text{otherwise } 
                                      \end{array}\right.
  \end{array}\] 
  In other words $\mu(\cdot)$ conflates all the non-interesting values to $\bot$. 
  Now we may consider the {\em set} which is the image of $D_\infty$ through $\mu$: 
  \[ D_\infty^\mu  \triangleq \mu(D_\infty) \] 

  Notice that this set is {\em finite} with cardinality at most $card(M) + 2$. Also, 
  we treat this is a {\em set}. Although $D_\infty$ has a domain structure, we do not 
  care about $D_\infty^\mu$ being a domain. 

  Now, in this $D_\infty^\mu$ we may redefine the interpretation of first-order constants
  and variable symbols in our theories, using ${\cal I}^\mu$ below:


  {\setlength{\arraycolsep}{2pt}  
  \[\begin{array}{rcl}
     \mlinterp{f_{ptr}} & = & \mu(\dbrace{P}^{\infty}(f)) \\  
 %% \roll(\ret(\inj{->}(\dlambda d_1 @.@ \ldots  \\
 %%                       &   & \quad \roll(\ret(\inj{->}(\dlambda d_n @.@ \\ 
 %%                       &   & \quad\quad\text{ if there exist } \oln{e}{n} \text{ s.t. } f[\taus]\;\ol{e} \in M \\ 
 %%                       &   & \quad\quad\quad\text{ and } \interp{e_i}{\dbrace{P}^\infty}{\cdot} = d_i\text{ then } \\
 %%                       &   & \quad\quad\quad\quad \mu(\dapp(\dbrace{P}^{\infty}(f),\oln{d}{n})) \\ 
 %%                       &   & \quad\quad\text{ else } \bot)))\ldots))) \\ \\ 
   \mlinterp{f^{n}}  & = & \dlambda (d {:} \prod_{n}D_{\infty}^\mu) @.@  \\
                       %% &   & \quad\quad\text{ if there exist } \oln{e}{n} \text{ s.t. } f[\taus]\;\ol{e} \in M \\ 
                       %% &   & \quad\quad\quad\text{ and } \interp{e_i}{\dbrace{P}^\infty}{\cdot} = \pi_i(d)\text{ then } \\
                       &   & \quad\quad (\mu\cdot\dapp)(\mu(\dbrace{P}^{\infty}(f)),\oln{\pi_i(d)}{i \in 1..n})) \\
                       %% &   & \quad\quad\text{ else } \bot \\ \\ 

   \mlinterp{app}     & = & \dlambda (d {:} D_{\infty}^\mu \times D_{\infty}^\mu) @.@ \\ \
                      &   & \quad\qquad \mu(\dapp(\pi_1(d),\pi_2(d))) \\ \\

   \mlinterp{K^{\ar}}     & = & \dlambda (d {:} \prod_{\ar}D_{\infty}^\mu) @.@ \mu(\roll(\ret(\inj{K}(d)))) \\ 
   \mlinterp{\sel{K}{i}} & = & \dlambda (d {:} D_{\infty}^{\mu}) @.@ \mu(\roll(\bind_g(\unroll(d)))) \\ 
     \text{where } g  & = & [\;\bot \\ 
                      &   & ,\;\dlambda d @.@ \unroll(\pi_i(d))  \quad (\text{case for constr. } K) \\ 
                      &   & ,\;\bot \\ 
                      &   & ,\;\ldots\\ 
                      &   & ,\;\bot\; ]
  \end{array}\]}

  Sadly, while the interpretation above is relatively simple, it does not validate the axiom 
  for \textsc{TUCase}. The fact that the denotation of a function application may be in the minimal set, 
  does not guarrantee that evaluation had proceeded along this function and hence the case scrutinee will
  be in the minimal set. This will be true only if we add an intentional test in the interpretation of 
  functions that queries the set $M$. {\bf DV:TODO tomorrow}. 

  {\bf DV: TODO: We need something like the definition below (but not quite, it does not type check yet)}: 
  {\setlength{\arraycolsep}{2pt}  
  \[\begin{array}{rcl}
     \mlinterp{f_{ptr}} & = & \mu(\roll(\ret(\inj{->}(\dlambda d_1 @.@ \ldots  \\
                       &   & \quad \mu(\roll(\ret(\inj{->}(\dlambda d_n @.@ \\ 
                       &   & \quad\quad\text{ if there exist } \oln{e}{n} \text{ s.t. } f[\taus]\;\ol{e} \in M \\ 
                       &   & \quad\quad\quad\text{ and } \interp{e_i}{\dbrace{P}^\infty}{\cdot} = d_i\text{ then } \\
                       &   & \quad\quad\quad\quad \mu(\dapp(\dbrace{P}^{\infty}(f),\oln{d}{n})) \\ 
                       &   & \quad\quad\text{ else } \bot))))\ldots)))) \\ \\ 
   \mlinterp{f^{n}}  & = & \dlambda (d {:} \prod_{n}D_{\infty}^\mu) @.@  \\
                       &   & \quad\quad\text{ if there exist } \oln{e}{n} \text{ s.t. } f[\taus]\;\ol{e} \in M \\ 
                       &   & \quad\quad\quad\text{ and } \interp{e_i}{\dbrace{P}^\infty}{\cdot} = \pi_i(d)\text{ then } \\
                       &   & \quad\quad\quad\quad \mu(\dapp(\dbrace{P}^{\infty}(f),\oln{\pi_i(d)}{i \in 1..n})) \\
                       &   & \quad\quad\text{ else } \bot \\ \\ 

   \mlinterp{app}     & = & \dlambda (d {:} D_{\infty}^\mu \times D_{\infty}^\mu) @.@ \\ \
                      &   & \quad\qquad \mu(\dapp(\pi_1(d),\pi_2(d))) \\ \\

   \mlinterp{K^{\ar}}     & = & \dlambda (d {:} \prod_{\ar}D_{\infty}^\mu) @.@ \mu(\roll(\ret(\inj{K}(d)))) \\ 
   \mlinterp{\sel{K}{i}} & = & \dlambda (d {:} D_{\infty}^{\mu}) @.@ \mu(\roll(\bind_g(\unroll(d)))) \\ 
     \text{where } g  & = & [\;\bot \\ 
                      &   & ,\;\dlambda d @.@ \unroll(\pi_i(d))  \quad (\text{case for constr. } K) \\ 
                      &   & ,\;\bot \\ 
                      &   & ,\;\ldots\\ 
                      &   & ,\;\bot\; ]
  \end{array}\]}

  \item The other case is when $@B@ = \CF$. {\bf TODO}
\end{itemize}







%% \newpage

%% \section{Contract checking soundness} 

%% \section{Contracts}

%% The syntax that we use for contracts is in Figure~\ref{fig:contract-syntax}. 
%% Contracts are typed (here, just monomorphically), and we give an operational 
%% semantics for contract satisfaction in the same figure. 

%% \begin{figure*}\small
%% \[\begin{array}{c} 
%% \ruleform{\Sigma;\Gamma |- \Ct } \\ \\ 
%% \prooftree
%% \Sigma;\Delta,x{:}\tau |- e : \Bool
%% ---------------------------------------{TCBase}
%% \Sigma;\Delta |- \{ (x{:}\tau) \mid e \} : \tau
%% ~~~~ 
%% \begin{array}{c}
%% \Sigma;\Delta |- \Ct_1 : \tau \\
%% \Sigma;\Delta,(x{:}\tau) |- \Ct_2 : \tau' 
%% \end{array}
%% ---------------------------------------{TCArr}
%% \Sigma;\Delta |- (x{:}\Ct_1) -> \Ct_2 : \tau -> \tau'
%% ~~~~ 
%% \Sigma;\Delta |- \Ct_1 : \tau \quad \Sigma;\Delta |- \Ct_2 : \tau 
%% ---------------------------------------{TCConj}
%% \Sigma;\Delta |- \Ct_1 \& \Ct_2 : \tau
%% ~~~~ 
%% \phantom{\Gamma}
%% ---------------------------------------{TCf}
%% \Sigma;\Delta |- \CF : \tau
%% \endprooftree \\ \\ 
%% \ruleform{\Sigma;P |- e \in \Ct} \\ \\
%% \prooftree
%%  P \not|- e \Downarrow 
%% -----------------------------------------------{ECDiv}
%%  \Sigma;P |- e \in \{ (x{:}\tau) \mid e' \}
%%  ~~~~
%%  P |- e'[e/x] \Downarrow \True 
%% -------------------------------------------{ECTrue}
%%  \Sigma;P |- e \in \{ (x{:}\tau) \mid e' \}
%%  ~~~~
%%  P \not|- e'[e/x] \Downarrow 
%%  ------------------------------------------{ECCDiv}
%%  \Sigma;P |- e \in \{ (x{:}\tau) \mid e' \} 
%%  ~~~~~
%%  \begin{array}{c} 
%%  \Sigma;\cdot |- \Ct_1 : \tau \\
%%  \text{for all } u, \Sigma;\cdot |- u : \tau ==> \Sigma;P |- e\;u \in \Ct_2[u/x]
%%  \end{array}
%%  --------------------------------------------{ECArr}
%%  \Sigma;P |- e \in (x{:}\Ct_1) -> \Ct_2 
%%  ~~~~
%%  \begin{array}{c}
%%  \Sigma,\cdot |- e : \tau  \\ 
%%  e \in \Ecf \quad \text{(See Section~\ref{sect:cf})}
%%  %% \text{for all } u, (\Sigma;\cdot |- u : \tau -> \Bool) /\ (@BAD@ \notin u) ==> \neg (P |- u\;e \Downarrow @BAD@)
%%  \end{array}
%%  --------------------------------------------------------------------------------------------{ECf}
%%  \Sigma;P |- e \in \CF 
%%  ~~~~~ 
%%  \Sigma;P |- e \in \Ct_1 \quad \Sigma;P |- e \in \Ct_2
%%  --------------------------------------------------------------------------------------------{ECConj}
%%  \Sigma;P |- e \in \Ct_1 \& \Ct_2
%% \endprooftree
%% \end{array}\]
%% \caption{Contract syntax and semantics}\label{fig:contract-syntax}
%% \end{figure*}







%% \section{Induction and admissibility}
%% {\bf TODO} 


%% \section{Minimization}
%% {\bf TODO} 

%% \section{Some ideas}
%% Sometimes the $\CF$ contract stands in our way e.g. for library stuff. It might 
%% be interesting to explore some user-defined pragma to side-step the $\bad$ case
%% in some pattern matches (i.e. make it on demand, pretty much as $F^{\star}$ does, where
%% only the user's assertions matter.
%% %% \acks
%% %% Acknowledgements here

\end{document}
